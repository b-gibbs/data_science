
% Default to the notebook output style

    


% Inherit from the specified cell style.




    
\documentclass[11pt]{article}

    
    
    \usepackage[T1]{fontenc}
    % Nicer default font (+ math font) than Computer Modern for most use cases
    \usepackage{mathpazo}

    % Basic figure setup, for now with no caption control since it's done
    % automatically by Pandoc (which extracts ![](path) syntax from Markdown).
    \usepackage{graphicx}
    % We will generate all images so they have a width \maxwidth. This means
    % that they will get their normal width if they fit onto the page, but
    % are scaled down if they would overflow the margins.
    \makeatletter
    \def\maxwidth{\ifdim\Gin@nat@width>\linewidth\linewidth
    \else\Gin@nat@width\fi}
    \makeatother
    \let\Oldincludegraphics\includegraphics
    % Set max figure width to be 80% of text width, for now hardcoded.
    \renewcommand{\includegraphics}[1]{\Oldincludegraphics[width=.8\maxwidth]{#1}}
    % Ensure that by default, figures have no caption (until we provide a
    % proper Figure object with a Caption API and a way to capture that
    % in the conversion process - todo).
    \usepackage{caption}
    \DeclareCaptionLabelFormat{nolabel}{}
    \captionsetup{labelformat=nolabel}

    \usepackage{adjustbox} % Used to constrain images to a maximum size 
    \usepackage{xcolor} % Allow colors to be defined
    \usepackage{enumerate} % Needed for markdown enumerations to work
    \usepackage{geometry} % Used to adjust the document margins
    \usepackage{amsmath} % Equations
    \usepackage{amssymb} % Equations
    \usepackage{textcomp} % defines textquotesingle
    % Hack from http://tex.stackexchange.com/a/47451/13684:
    \AtBeginDocument{%
        \def\PYZsq{\textquotesingle}% Upright quotes in Pygmentized code
    }
    \usepackage{upquote} % Upright quotes for verbatim code
    \usepackage{eurosym} % defines \euro
    \usepackage[mathletters]{ucs} % Extended unicode (utf-8) support
    \usepackage[utf8x]{inputenc} % Allow utf-8 characters in the tex document
    \usepackage{fancyvrb} % verbatim replacement that allows latex
    \usepackage{grffile} % extends the file name processing of package graphics 
                         % to support a larger range 
    % The hyperref package gives us a pdf with properly built
    % internal navigation ('pdf bookmarks' for the table of contents,
    % internal cross-reference links, web links for URLs, etc.)
    \usepackage{hyperref}
    \usepackage{longtable} % longtable support required by pandoc >1.10
    \usepackage{booktabs}  % table support for pandoc > 1.12.2
    \usepackage[inline]{enumitem} % IRkernel/repr support (it uses the enumerate* environment)
    \usepackage[normalem]{ulem} % ulem is needed to support strikethroughs (\sout)
                                % normalem makes italics be italics, not underlines
    

    
    
    % Colors for the hyperref package
    \definecolor{urlcolor}{rgb}{0,.145,.698}
    \definecolor{linkcolor}{rgb}{.71,0.21,0.01}
    \definecolor{citecolor}{rgb}{.12,.54,.11}

    % ANSI colors
    \definecolor{ansi-black}{HTML}{3E424D}
    \definecolor{ansi-black-intense}{HTML}{282C36}
    \definecolor{ansi-red}{HTML}{E75C58}
    \definecolor{ansi-red-intense}{HTML}{B22B31}
    \definecolor{ansi-green}{HTML}{00A250}
    \definecolor{ansi-green-intense}{HTML}{007427}
    \definecolor{ansi-yellow}{HTML}{DDB62B}
    \definecolor{ansi-yellow-intense}{HTML}{B27D12}
    \definecolor{ansi-blue}{HTML}{208FFB}
    \definecolor{ansi-blue-intense}{HTML}{0065CA}
    \definecolor{ansi-magenta}{HTML}{D160C4}
    \definecolor{ansi-magenta-intense}{HTML}{A03196}
    \definecolor{ansi-cyan}{HTML}{60C6C8}
    \definecolor{ansi-cyan-intense}{HTML}{258F8F}
    \definecolor{ansi-white}{HTML}{C5C1B4}
    \definecolor{ansi-white-intense}{HTML}{A1A6B2}

    % commands and environments needed by pandoc snippets
    % extracted from the output of `pandoc -s`
    \providecommand{\tightlist}{%
      \setlength{\itemsep}{0pt}\setlength{\parskip}{0pt}}
    \DefineVerbatimEnvironment{Highlighting}{Verbatim}{commandchars=\\\{\}}
    % Add ',fontsize=\small' for more characters per line
    \newenvironment{Shaded}{}{}
    \newcommand{\KeywordTok}[1]{\textcolor[rgb]{0.00,0.44,0.13}{\textbf{{#1}}}}
    \newcommand{\DataTypeTok}[1]{\textcolor[rgb]{0.56,0.13,0.00}{{#1}}}
    \newcommand{\DecValTok}[1]{\textcolor[rgb]{0.25,0.63,0.44}{{#1}}}
    \newcommand{\BaseNTok}[1]{\textcolor[rgb]{0.25,0.63,0.44}{{#1}}}
    \newcommand{\FloatTok}[1]{\textcolor[rgb]{0.25,0.63,0.44}{{#1}}}
    \newcommand{\CharTok}[1]{\textcolor[rgb]{0.25,0.44,0.63}{{#1}}}
    \newcommand{\StringTok}[1]{\textcolor[rgb]{0.25,0.44,0.63}{{#1}}}
    \newcommand{\CommentTok}[1]{\textcolor[rgb]{0.38,0.63,0.69}{\textit{{#1}}}}
    \newcommand{\OtherTok}[1]{\textcolor[rgb]{0.00,0.44,0.13}{{#1}}}
    \newcommand{\AlertTok}[1]{\textcolor[rgb]{1.00,0.00,0.00}{\textbf{{#1}}}}
    \newcommand{\FunctionTok}[1]{\textcolor[rgb]{0.02,0.16,0.49}{{#1}}}
    \newcommand{\RegionMarkerTok}[1]{{#1}}
    \newcommand{\ErrorTok}[1]{\textcolor[rgb]{1.00,0.00,0.00}{\textbf{{#1}}}}
    \newcommand{\NormalTok}[1]{{#1}}
    
    % Additional commands for more recent versions of Pandoc
    \newcommand{\ConstantTok}[1]{\textcolor[rgb]{0.53,0.00,0.00}{{#1}}}
    \newcommand{\SpecialCharTok}[1]{\textcolor[rgb]{0.25,0.44,0.63}{{#1}}}
    \newcommand{\VerbatimStringTok}[1]{\textcolor[rgb]{0.25,0.44,0.63}{{#1}}}
    \newcommand{\SpecialStringTok}[1]{\textcolor[rgb]{0.73,0.40,0.53}{{#1}}}
    \newcommand{\ImportTok}[1]{{#1}}
    \newcommand{\DocumentationTok}[1]{\textcolor[rgb]{0.73,0.13,0.13}{\textit{{#1}}}}
    \newcommand{\AnnotationTok}[1]{\textcolor[rgb]{0.38,0.63,0.69}{\textbf{\textit{{#1}}}}}
    \newcommand{\CommentVarTok}[1]{\textcolor[rgb]{0.38,0.63,0.69}{\textbf{\textit{{#1}}}}}
    \newcommand{\VariableTok}[1]{\textcolor[rgb]{0.10,0.09,0.49}{{#1}}}
    \newcommand{\ControlFlowTok}[1]{\textcolor[rgb]{0.00,0.44,0.13}{\textbf{{#1}}}}
    \newcommand{\OperatorTok}[1]{\textcolor[rgb]{0.40,0.40,0.40}{{#1}}}
    \newcommand{\BuiltInTok}[1]{{#1}}
    \newcommand{\ExtensionTok}[1]{{#1}}
    \newcommand{\PreprocessorTok}[1]{\textcolor[rgb]{0.74,0.48,0.00}{{#1}}}
    \newcommand{\AttributeTok}[1]{\textcolor[rgb]{0.49,0.56,0.16}{{#1}}}
    \newcommand{\InformationTok}[1]{\textcolor[rgb]{0.38,0.63,0.69}{\textbf{\textit{{#1}}}}}
    \newcommand{\WarningTok}[1]{\textcolor[rgb]{0.38,0.63,0.69}{\textbf{\textit{{#1}}}}}
    
    
    % Define a nice break command that doesn't care if a line doesn't already
    % exist.
    \def\br{\hspace*{\fill} \\* }
    % Math Jax compatability definitions
    \def\gt{>}
    \def\lt{<}
    % Document parameters
    \title{sliderule\_dsi\_inferential\_statistics\_exercise\_1}
    
    
    

    % Pygments definitions
    
\makeatletter
\def\PY@reset{\let\PY@it=\relax \let\PY@bf=\relax%
    \let\PY@ul=\relax \let\PY@tc=\relax%
    \let\PY@bc=\relax \let\PY@ff=\relax}
\def\PY@tok#1{\csname PY@tok@#1\endcsname}
\def\PY@toks#1+{\ifx\relax#1\empty\else%
    \PY@tok{#1}\expandafter\PY@toks\fi}
\def\PY@do#1{\PY@bc{\PY@tc{\PY@ul{%
    \PY@it{\PY@bf{\PY@ff{#1}}}}}}}
\def\PY#1#2{\PY@reset\PY@toks#1+\relax+\PY@do{#2}}

\expandafter\def\csname PY@tok@w\endcsname{\def\PY@tc##1{\textcolor[rgb]{0.73,0.73,0.73}{##1}}}
\expandafter\def\csname PY@tok@c\endcsname{\let\PY@it=\textit\def\PY@tc##1{\textcolor[rgb]{0.25,0.50,0.50}{##1}}}
\expandafter\def\csname PY@tok@cp\endcsname{\def\PY@tc##1{\textcolor[rgb]{0.74,0.48,0.00}{##1}}}
\expandafter\def\csname PY@tok@k\endcsname{\let\PY@bf=\textbf\def\PY@tc##1{\textcolor[rgb]{0.00,0.50,0.00}{##1}}}
\expandafter\def\csname PY@tok@kp\endcsname{\def\PY@tc##1{\textcolor[rgb]{0.00,0.50,0.00}{##1}}}
\expandafter\def\csname PY@tok@kt\endcsname{\def\PY@tc##1{\textcolor[rgb]{0.69,0.00,0.25}{##1}}}
\expandafter\def\csname PY@tok@o\endcsname{\def\PY@tc##1{\textcolor[rgb]{0.40,0.40,0.40}{##1}}}
\expandafter\def\csname PY@tok@ow\endcsname{\let\PY@bf=\textbf\def\PY@tc##1{\textcolor[rgb]{0.67,0.13,1.00}{##1}}}
\expandafter\def\csname PY@tok@nb\endcsname{\def\PY@tc##1{\textcolor[rgb]{0.00,0.50,0.00}{##1}}}
\expandafter\def\csname PY@tok@nf\endcsname{\def\PY@tc##1{\textcolor[rgb]{0.00,0.00,1.00}{##1}}}
\expandafter\def\csname PY@tok@nc\endcsname{\let\PY@bf=\textbf\def\PY@tc##1{\textcolor[rgb]{0.00,0.00,1.00}{##1}}}
\expandafter\def\csname PY@tok@nn\endcsname{\let\PY@bf=\textbf\def\PY@tc##1{\textcolor[rgb]{0.00,0.00,1.00}{##1}}}
\expandafter\def\csname PY@tok@ne\endcsname{\let\PY@bf=\textbf\def\PY@tc##1{\textcolor[rgb]{0.82,0.25,0.23}{##1}}}
\expandafter\def\csname PY@tok@nv\endcsname{\def\PY@tc##1{\textcolor[rgb]{0.10,0.09,0.49}{##1}}}
\expandafter\def\csname PY@tok@no\endcsname{\def\PY@tc##1{\textcolor[rgb]{0.53,0.00,0.00}{##1}}}
\expandafter\def\csname PY@tok@nl\endcsname{\def\PY@tc##1{\textcolor[rgb]{0.63,0.63,0.00}{##1}}}
\expandafter\def\csname PY@tok@ni\endcsname{\let\PY@bf=\textbf\def\PY@tc##1{\textcolor[rgb]{0.60,0.60,0.60}{##1}}}
\expandafter\def\csname PY@tok@na\endcsname{\def\PY@tc##1{\textcolor[rgb]{0.49,0.56,0.16}{##1}}}
\expandafter\def\csname PY@tok@nt\endcsname{\let\PY@bf=\textbf\def\PY@tc##1{\textcolor[rgb]{0.00,0.50,0.00}{##1}}}
\expandafter\def\csname PY@tok@nd\endcsname{\def\PY@tc##1{\textcolor[rgb]{0.67,0.13,1.00}{##1}}}
\expandafter\def\csname PY@tok@s\endcsname{\def\PY@tc##1{\textcolor[rgb]{0.73,0.13,0.13}{##1}}}
\expandafter\def\csname PY@tok@sd\endcsname{\let\PY@it=\textit\def\PY@tc##1{\textcolor[rgb]{0.73,0.13,0.13}{##1}}}
\expandafter\def\csname PY@tok@si\endcsname{\let\PY@bf=\textbf\def\PY@tc##1{\textcolor[rgb]{0.73,0.40,0.53}{##1}}}
\expandafter\def\csname PY@tok@se\endcsname{\let\PY@bf=\textbf\def\PY@tc##1{\textcolor[rgb]{0.73,0.40,0.13}{##1}}}
\expandafter\def\csname PY@tok@sr\endcsname{\def\PY@tc##1{\textcolor[rgb]{0.73,0.40,0.53}{##1}}}
\expandafter\def\csname PY@tok@ss\endcsname{\def\PY@tc##1{\textcolor[rgb]{0.10,0.09,0.49}{##1}}}
\expandafter\def\csname PY@tok@sx\endcsname{\def\PY@tc##1{\textcolor[rgb]{0.00,0.50,0.00}{##1}}}
\expandafter\def\csname PY@tok@m\endcsname{\def\PY@tc##1{\textcolor[rgb]{0.40,0.40,0.40}{##1}}}
\expandafter\def\csname PY@tok@gh\endcsname{\let\PY@bf=\textbf\def\PY@tc##1{\textcolor[rgb]{0.00,0.00,0.50}{##1}}}
\expandafter\def\csname PY@tok@gu\endcsname{\let\PY@bf=\textbf\def\PY@tc##1{\textcolor[rgb]{0.50,0.00,0.50}{##1}}}
\expandafter\def\csname PY@tok@gd\endcsname{\def\PY@tc##1{\textcolor[rgb]{0.63,0.00,0.00}{##1}}}
\expandafter\def\csname PY@tok@gi\endcsname{\def\PY@tc##1{\textcolor[rgb]{0.00,0.63,0.00}{##1}}}
\expandafter\def\csname PY@tok@gr\endcsname{\def\PY@tc##1{\textcolor[rgb]{1.00,0.00,0.00}{##1}}}
\expandafter\def\csname PY@tok@ge\endcsname{\let\PY@it=\textit}
\expandafter\def\csname PY@tok@gs\endcsname{\let\PY@bf=\textbf}
\expandafter\def\csname PY@tok@gp\endcsname{\let\PY@bf=\textbf\def\PY@tc##1{\textcolor[rgb]{0.00,0.00,0.50}{##1}}}
\expandafter\def\csname PY@tok@go\endcsname{\def\PY@tc##1{\textcolor[rgb]{0.53,0.53,0.53}{##1}}}
\expandafter\def\csname PY@tok@gt\endcsname{\def\PY@tc##1{\textcolor[rgb]{0.00,0.27,0.87}{##1}}}
\expandafter\def\csname PY@tok@err\endcsname{\def\PY@bc##1{\setlength{\fboxsep}{0pt}\fcolorbox[rgb]{1.00,0.00,0.00}{1,1,1}{\strut ##1}}}
\expandafter\def\csname PY@tok@kc\endcsname{\let\PY@bf=\textbf\def\PY@tc##1{\textcolor[rgb]{0.00,0.50,0.00}{##1}}}
\expandafter\def\csname PY@tok@kd\endcsname{\let\PY@bf=\textbf\def\PY@tc##1{\textcolor[rgb]{0.00,0.50,0.00}{##1}}}
\expandafter\def\csname PY@tok@kn\endcsname{\let\PY@bf=\textbf\def\PY@tc##1{\textcolor[rgb]{0.00,0.50,0.00}{##1}}}
\expandafter\def\csname PY@tok@kr\endcsname{\let\PY@bf=\textbf\def\PY@tc##1{\textcolor[rgb]{0.00,0.50,0.00}{##1}}}
\expandafter\def\csname PY@tok@bp\endcsname{\def\PY@tc##1{\textcolor[rgb]{0.00,0.50,0.00}{##1}}}
\expandafter\def\csname PY@tok@fm\endcsname{\def\PY@tc##1{\textcolor[rgb]{0.00,0.00,1.00}{##1}}}
\expandafter\def\csname PY@tok@vc\endcsname{\def\PY@tc##1{\textcolor[rgb]{0.10,0.09,0.49}{##1}}}
\expandafter\def\csname PY@tok@vg\endcsname{\def\PY@tc##1{\textcolor[rgb]{0.10,0.09,0.49}{##1}}}
\expandafter\def\csname PY@tok@vi\endcsname{\def\PY@tc##1{\textcolor[rgb]{0.10,0.09,0.49}{##1}}}
\expandafter\def\csname PY@tok@vm\endcsname{\def\PY@tc##1{\textcolor[rgb]{0.10,0.09,0.49}{##1}}}
\expandafter\def\csname PY@tok@sa\endcsname{\def\PY@tc##1{\textcolor[rgb]{0.73,0.13,0.13}{##1}}}
\expandafter\def\csname PY@tok@sb\endcsname{\def\PY@tc##1{\textcolor[rgb]{0.73,0.13,0.13}{##1}}}
\expandafter\def\csname PY@tok@sc\endcsname{\def\PY@tc##1{\textcolor[rgb]{0.73,0.13,0.13}{##1}}}
\expandafter\def\csname PY@tok@dl\endcsname{\def\PY@tc##1{\textcolor[rgb]{0.73,0.13,0.13}{##1}}}
\expandafter\def\csname PY@tok@s2\endcsname{\def\PY@tc##1{\textcolor[rgb]{0.73,0.13,0.13}{##1}}}
\expandafter\def\csname PY@tok@sh\endcsname{\def\PY@tc##1{\textcolor[rgb]{0.73,0.13,0.13}{##1}}}
\expandafter\def\csname PY@tok@s1\endcsname{\def\PY@tc##1{\textcolor[rgb]{0.73,0.13,0.13}{##1}}}
\expandafter\def\csname PY@tok@mb\endcsname{\def\PY@tc##1{\textcolor[rgb]{0.40,0.40,0.40}{##1}}}
\expandafter\def\csname PY@tok@mf\endcsname{\def\PY@tc##1{\textcolor[rgb]{0.40,0.40,0.40}{##1}}}
\expandafter\def\csname PY@tok@mh\endcsname{\def\PY@tc##1{\textcolor[rgb]{0.40,0.40,0.40}{##1}}}
\expandafter\def\csname PY@tok@mi\endcsname{\def\PY@tc##1{\textcolor[rgb]{0.40,0.40,0.40}{##1}}}
\expandafter\def\csname PY@tok@il\endcsname{\def\PY@tc##1{\textcolor[rgb]{0.40,0.40,0.40}{##1}}}
\expandafter\def\csname PY@tok@mo\endcsname{\def\PY@tc##1{\textcolor[rgb]{0.40,0.40,0.40}{##1}}}
\expandafter\def\csname PY@tok@ch\endcsname{\let\PY@it=\textit\def\PY@tc##1{\textcolor[rgb]{0.25,0.50,0.50}{##1}}}
\expandafter\def\csname PY@tok@cm\endcsname{\let\PY@it=\textit\def\PY@tc##1{\textcolor[rgb]{0.25,0.50,0.50}{##1}}}
\expandafter\def\csname PY@tok@cpf\endcsname{\let\PY@it=\textit\def\PY@tc##1{\textcolor[rgb]{0.25,0.50,0.50}{##1}}}
\expandafter\def\csname PY@tok@c1\endcsname{\let\PY@it=\textit\def\PY@tc##1{\textcolor[rgb]{0.25,0.50,0.50}{##1}}}
\expandafter\def\csname PY@tok@cs\endcsname{\let\PY@it=\textit\def\PY@tc##1{\textcolor[rgb]{0.25,0.50,0.50}{##1}}}

\def\PYZbs{\char`\\}
\def\PYZus{\char`\_}
\def\PYZob{\char`\{}
\def\PYZcb{\char`\}}
\def\PYZca{\char`\^}
\def\PYZam{\char`\&}
\def\PYZlt{\char`\<}
\def\PYZgt{\char`\>}
\def\PYZsh{\char`\#}
\def\PYZpc{\char`\%}
\def\PYZdl{\char`\$}
\def\PYZhy{\char`\-}
\def\PYZsq{\char`\'}
\def\PYZdq{\char`\"}
\def\PYZti{\char`\~}
% for compatibility with earlier versions
\def\PYZat{@}
\def\PYZlb{[}
\def\PYZrb{]}
\makeatother


    % Exact colors from NB
    \definecolor{incolor}{rgb}{0.0, 0.0, 0.5}
    \definecolor{outcolor}{rgb}{0.545, 0.0, 0.0}



    
    % Prevent overflowing lines due to hard-to-break entities
    \sloppy 
    % Setup hyperref package
    \hypersetup{
      breaklinks=true,  % so long urls are correctly broken across lines
      colorlinks=true,
      urlcolor=urlcolor,
      linkcolor=linkcolor,
      citecolor=citecolor,
      }
    % Slightly bigger margins than the latex defaults
    
    \geometry{verbose,tmargin=1in,bmargin=1in,lmargin=1in,rmargin=1in}
    
    

    \begin{document}
    
    
    \maketitle
    
    

    
    \section{What is the True Normal Human Body
Temperature?}\label{what-is-the-true-normal-human-body-temperature}

    \paragraph{Background}\label{background}

The mean normal body temperature was held to be 37\(^{\circ}\)C or
98.6\(^{\circ}\)F for more than 120 years since it was first
conceptualized and reported by Carl Wunderlich in a famous 1868 book.
But, is this value statistically correct?

    Exercises

In this exercise, you will analyze a dataset of human body temperatures
and employ the concepts of hypothesis testing, confidence intervals, and
statistical significance.

Answer the following questions in this notebook below and submit to your
Github account.

Is the distribution of body temperatures normal?

\begin{verbatim}
<li> Although this is not a requirement for the Central Limit Theorem to hold (read the introduction on Wikipedia's page about the CLT carefully: https://en.wikipedia.org/wiki/Central_limit_theorem), it gives us some peace of mind that the population may also be normally distributed if we assume that this sample is representative of the population.
<li> Think about the way you're going to check for the normality of the distribution. Graphical methods are usually used first, but there are also other ways: https://en.wikipedia.org/wiki/Normality_test
</ul>
\end{verbatim}

Is the sample size large? Are the observations independent?

\begin{verbatim}
<li> Remember that this is a condition for the Central Limit Theorem, and hence the statistical tests we are using, to apply.
</ul>
\end{verbatim}

Is the true population mean really 98.6 degrees F?

\begin{verbatim}
<li> First, try a bootstrap hypothesis test.
<li> Now, let's try frequentist statistical testing. Would you use a one-sample or two-sample test? Why?
<li> In this situation, is it appropriate to use the $t$ or $z$ statistic? 
<li> Now try using the other test. How is the result be different? Why?
</ul>
\end{verbatim}

Draw a small sample of size 10 from the data and repeat both frequentist
tests.

\begin{verbatim}
<li> Which one is the correct one to use? 
<li> What do you notice? What does this tell you about the difference in application of the $t$ and $z$ statistic?
</ul>
\end{verbatim}

At what temperature should we consider someone's temperature to be
"abnormal"?

\begin{verbatim}
<li> As in the previous example, try calculating everything using the boostrap approach, as well as the frequentist approach.
<li> Start by computing the margin of error and confidence interval. When calculating the confidence interval, keep in mind that you should use the appropriate formula for one draw, and not N draws.
</ul>
\end{verbatim}

Is there a significant difference between males and females in normal
temperature?

\begin{verbatim}
<li> What testing approach did you use and why?
<li> Write a story with your conclusion in the context of the original problem.
</ul>
\end{verbatim}

You can include written notes in notebook cells using Markdown: - In the
control panel at the top, choose Cell \textgreater{} Cell Type
\textgreater{} Markdown - Markdown syntax:
http://nestacms.com/docs/creating-content/markdown-cheat-sheet

\paragraph{Resources}\label{resources}

\begin{itemize}
\tightlist
\item
  Information and data sources:
  http://www.amstat.org/publications/jse/datasets/normtemp.txt,
  http://www.amstat.org/publications/jse/jse\_data\_archive.htm
\item
  Markdown syntax:
  http://nestacms.com/docs/creating-content/markdown-cheat-sheet
\end{itemize}

\begin{center}\rule{0.5\linewidth}{\linethickness}\end{center}

    \subsection{Summary of Findings}\label{summary-of-findings}

\begin{center}\rule{0.5\linewidth}{\linethickness}\end{center}

The mean normal body temperature was held to be 98.6°F for more than 120
years since it was first conceptualized and reported by Carl Wunderlich
in a famous 1868 book. After examining a data set of 130 samples (65
males and 65 females), we'll see that the mean normal body temperature
is actually 98.25°F. We'll also see that the mean body temperature of
females is significantly higher than males'.

    \begin{Verbatim}[commandchars=\\\{\}]
{\color{incolor}In [{\color{incolor}1}]:} \PY{c+c1}{\PYZsh{} IMPORTS}
        \PY{k+kn}{import} \PY{n+nn}{numpy} \PY{k}{as} \PY{n+nn}{np}
        \PY{k+kn}{import} \PY{n+nn}{pandas} \PY{k}{as} \PY{n+nn}{pd}
        \PY{k+kn}{import} \PY{n+nn}{matplotlib}\PY{n+nn}{.}\PY{n+nn}{pyplot} \PY{k}{as} \PY{n+nn}{plt}
        \PY{k+kn}{import} \PY{n+nn}{seaborn} \PY{k}{as} \PY{n+nn}{sns}
        \PY{k+kn}{import} \PY{n+nn}{scipy}\PY{n+nn}{.}\PY{n+nn}{stats} \PY{k}{as} \PY{n+nn}{stats}
        \PY{k+kn}{import} \PY{n+nn}{statsmodels}\PY{n+nn}{.}\PY{n+nn}{stats} \PY{k}{as} \PY{n+nn}{smd}
        \PY{k+kn}{import} \PY{n+nn}{pylab}
        
        \PY{c+c1}{\PYZsh{} matplotlib setup}
        \PY{o}{\PYZpc{}}\PY{k}{matplotlib} inline
        
        \PY{c+c1}{\PYZsh{} turn edges on in plt}
        \PY{n}{plt}\PY{o}{.}\PY{n}{rcParams}\PY{p}{[}\PY{l+s+s2}{\PYZdq{}}\PY{l+s+s2}{patch.force\PYZus{}edgecolor}\PY{l+s+s2}{\PYZdq{}}\PY{p}{]} \PY{o}{=} \PY{k+kc}{True}
        
        \PY{n}{np}\PY{o}{.}\PY{n}{random}\PY{o}{.}\PY{n}{seed}\PY{p}{(}\PY{l+m+mi}{42}\PY{p}{)}
        
        \PY{n}{df} \PY{o}{=} \PY{n}{pd}\PY{o}{.}\PY{n}{read\PYZus{}csv}\PY{p}{(}\PY{l+s+s1}{\PYZsq{}}\PY{l+s+s1}{data/human\PYZus{}body\PYZus{}temperature.csv}\PY{l+s+s1}{\PYZsq{}}\PY{p}{)}
        \PY{n}{df}\PY{o}{.}\PY{n}{head}\PY{p}{(}\PY{p}{)}
\end{Verbatim}


\begin{Verbatim}[commandchars=\\\{\}]
{\color{outcolor}Out[{\color{outcolor}1}]:}    temperature gender  heart\_rate
        0         99.3      F        68.0
        1         98.4      F        81.0
        2         97.8      M        73.0
        3         99.2      F        66.0
        4         98.0      F        73.0
\end{Verbatim}
            
    \begin{Verbatim}[commandchars=\\\{\}]
{\color{incolor}In [{\color{incolor}2}]:} \PY{c+c1}{\PYZsh{} FUNCTIONS USED}
        
        \PY{k}{def} \PY{n+nf}{ecdf}\PY{p}{(}\PY{n}{data}\PY{p}{)}\PY{p}{:}
            \PY{l+s+sd}{\PYZdq{}\PYZdq{}\PYZdq{}Compute ECDF for a one\PYZhy{}dimensional array of measurements.\PYZdq{}\PYZdq{}\PYZdq{}}
        
            \PY{c+c1}{\PYZsh{} Number of data points: n}
            \PY{n}{n} \PY{o}{=} \PY{n+nb}{len}\PY{p}{(}\PY{n}{data}\PY{p}{)}
        
            \PY{c+c1}{\PYZsh{} x: sort the data}
            \PY{n}{x} \PY{o}{=} \PY{n}{np}\PY{o}{.}\PY{n}{sort}\PY{p}{(}\PY{n}{data}\PY{p}{)}
        
            \PY{c+c1}{\PYZsh{} y: range for y\PYZhy{}axis}
            \PY{n}{y} \PY{o}{=} \PY{n}{np}\PY{o}{.}\PY{n}{arange}\PY{p}{(}\PY{l+m+mi}{1}\PY{p}{,} \PY{n}{n}\PY{o}{+}\PY{l+m+mi}{1}\PY{p}{)} \PY{o}{/} \PY{n}{n}
        
            \PY{k}{return} \PY{n}{x}\PY{p}{,} \PY{n}{y}
\end{Verbatim}


    \subsubsection{Question \#1}\label{question-1}

\begin{center}\rule{0.5\linewidth}{\linethickness}\end{center}

Is the distribution of body temperatures normal?\\

\begin{verbatim}
<li>Although this is not a requirement for the Central Limit Theorem to hold (read the introduction on Wikipedia's page about the CLT carefully: https://en.wikipedia.org/wiki/Central_limit_theorem), it gives us some peace of mind that the population may also be normally distributed if we assume that this sample is representative of the population.</br>
<li>Think about the way you're going to check for the normality of the distribution. Graphical methods are usually used first, but there are also other ways: https://en.wikipedia.org/wiki/Normality_test  
\end{verbatim}

    \begin{Verbatim}[commandchars=\\\{\}]
{\color{incolor}In [{\color{incolor}3}]:} \PY{c+c1}{\PYZsh{} Plot a histogram to see whether the distribution is normal}
        \PY{n}{temp} \PY{o}{=} \PY{n}{df}\PY{o}{.}\PY{n}{temperature}
        
        \PY{n}{sns}\PY{o}{.}\PY{n}{set}\PY{p}{(}\PY{n}{rc}\PY{o}{=}\PY{p}{\PYZob{}}\PY{l+s+s2}{\PYZdq{}}\PY{l+s+s2}{figure.figsize}\PY{l+s+s2}{\PYZdq{}}\PY{p}{:} \PY{p}{(}\PY{l+m+mi}{12}\PY{p}{,} \PY{l+m+mi}{8}\PY{p}{)}\PY{p}{\PYZcb{}}\PY{p}{)}
        \PY{n}{plt}\PY{o}{.}\PY{n}{style}\PY{o}{.}\PY{n}{use}\PY{p}{(}\PY{l+s+s1}{\PYZsq{}}\PY{l+s+s1}{fivethirtyeight}\PY{l+s+s1}{\PYZsq{}}\PY{p}{)}
        
        \PY{c+c1}{\PYZsh{} Number of bins is the square root of number of data points: n\PYZus{}bins}
        \PY{n}{n\PYZus{}bins} \PY{o}{=} \PY{n}{np}\PY{o}{.}\PY{n}{sqrt}\PY{p}{(}\PY{n+nb}{len}\PY{p}{(}\PY{n}{temp}\PY{p}{)}\PY{p}{)}
        
        \PY{c+c1}{\PYZsh{} Convert number of bins to integer: n\PYZus{}bins}
        \PY{n}{n\PYZus{}bins} \PY{o}{=} \PY{n+nb}{int}\PY{p}{(}\PY{n}{n\PYZus{}bins}\PY{p}{)}
        
        \PY{c+c1}{\PYZsh{} plot hist}
        \PY{n}{\PYZus{}} \PY{o}{=} \PY{n}{plt}\PY{o}{.}\PY{n}{hist}\PY{p}{(}\PY{n}{temp}\PY{p}{,} \PY{n}{density}\PY{o}{=}\PY{k+kc}{True}\PY{p}{,} \PY{n}{bins}\PY{o}{=}\PY{n}{n\PYZus{}bins}\PY{p}{)}
        
        \PY{c+c1}{\PYZsh{} overlay PDF of the Standard Normal Distribution}
        \PY{n}{x} \PY{o}{=} \PY{n}{np}\PY{o}{.}\PY{n}{linspace}\PY{p}{(}\PY{n}{np}\PY{o}{.}\PY{n}{min}\PY{p}{(}\PY{n}{temp}\PY{p}{)} \PY{o}{\PYZhy{}} \PY{l+m+mi}{1}\PY{p}{,} \PY{n}{np}\PY{o}{.}\PY{n}{max}\PY{p}{(}\PY{n}{temp}\PY{p}{)} \PY{o}{+} \PY{l+m+mi}{1}\PY{p}{,} \PY{l+m+mi}{100}\PY{p}{,} \PY{n}{endpoint}\PY{o}{=}\PY{k+kc}{True}\PY{p}{)}
        \PY{n}{pdf} \PY{o}{=} \PY{p}{[}\PY{n}{stats}\PY{o}{.}\PY{n}{norm}\PY{o}{.}\PY{n}{pdf}\PY{p}{(}\PY{n}{\PYZus{}}\PY{p}{,} \PY{n}{loc}\PY{o}{=}\PY{n}{np}\PY{o}{.}\PY{n}{mean}\PY{p}{(}\PY{n}{temp}\PY{p}{)}\PY{p}{,} \PY{n}{scale}\PY{o}{=}\PY{n}{np}\PY{o}{.}\PY{n}{std}\PY{p}{(}\PY{n}{temp}\PY{p}{)}\PY{p}{)} \PY{k}{for} \PY{n}{\PYZus{}} \PY{o+ow}{in} \PY{n}{x}\PY{p}{]}
        \PY{n}{plt}\PY{o}{.}\PY{n}{plot}\PY{p}{(}\PY{n}{x}\PY{p}{,} \PY{n}{pdf}\PY{p}{,} \PY{l+s+s1}{\PYZsq{}}\PY{l+s+s1}{k\PYZhy{}}\PY{l+s+s1}{\PYZsq{}}\PY{p}{)}
        
        
        \PY{c+c1}{\PYZsh{} labels}
        \PY{n}{\PYZus{}} \PY{o}{=} \PY{n}{plt}\PY{o}{.}\PY{n}{text}\PY{p}{(}\PY{l+m+mf}{95.5}\PY{p}{,} \PY{l+m+mf}{0.5}\PY{p}{,} \PY{l+s+sa}{r}\PY{l+s+s1}{\PYZsq{}}\PY{l+s+s1}{\PYZdl{}}\PY{l+s+s1}{\PYZbs{}}\PY{l+s+s1}{mu= }\PY{l+s+si}{\PYZob{}\PYZcb{}}\PY{l+s+s1}{,}\PY{l+s+s1}{\PYZbs{}}\PY{l+s+s1}{ }\PY{l+s+s1}{\PYZbs{}}\PY{l+s+s1}{sigma=}\PY{l+s+si}{\PYZob{}\PYZcb{}}\PY{l+s+s1}{\PYZdl{}}\PY{l+s+s1}{\PYZsq{}}\PY{o}{.}\PY{n}{format}\PY{p}{(}\PY{n+nb}{round}\PY{p}{(}\PY{n}{np}\PY{o}{.}\PY{n}{mean}\PY{p}{(}\PY{n}{temp}\PY{p}{)}\PY{p}{,} \PY{l+m+mi}{2}\PY{p}{)}\PY{p}{,} \PY{n+nb}{round}\PY{p}{(}\PY{n}{np}\PY{o}{.}\PY{n}{std}\PY{p}{(}\PY{n}{temp}\PY{p}{)}\PY{p}{,} \PY{l+m+mi}{2}\PY{p}{)}\PY{p}{)}\PY{p}{)}
        \PY{n}{\PYZus{}} \PY{o}{=} \PY{n}{plt}\PY{o}{.}\PY{n}{xlabel}\PY{p}{(}\PY{l+s+s1}{\PYZsq{}}\PY{l+s+s1}{Temperature (F)}\PY{l+s+s1}{\PYZsq{}}\PY{p}{)}
        \PY{n}{\PYZus{}} \PY{o}{=} \PY{n}{plt}\PY{o}{.}\PY{n}{ylabel}\PY{p}{(}\PY{l+s+s1}{\PYZsq{}}\PY{l+s+s1}{Frequency}\PY{l+s+s1}{\PYZsq{}}\PY{p}{)}
        \PY{n}{\PYZus{}} \PY{o}{=} \PY{n}{plt}\PY{o}{.}\PY{n}{title}\PY{p}{(}\PY{l+s+s1}{\PYZsq{}}\PY{l+s+s1}{Fig. 1.1: Human Body Temperatures}\PY{l+s+s1}{\PYZsq{}}\PY{p}{)}
        
        \PY{n}{margins} \PY{o}{=} \PY{l+m+mf}{0.02}
        \PY{n}{\PYZus{}} \PY{o}{=} \PY{n}{plt}\PY{o}{.}\PY{n}{legend}\PY{p}{(}\PY{p}{(}\PY{l+s+s1}{\PYZsq{}}\PY{l+s+s1}{Std Normal Dist}\PY{l+s+s1}{\PYZsq{}}\PY{p}{,} \PY{l+s+s1}{\PYZsq{}}\PY{l+s+s1}{Samples}\PY{l+s+s1}{\PYZsq{}}\PY{p}{)}\PY{p}{)}
        \PY{n}{plt}\PY{o}{.}\PY{n}{ylim}\PY{p}{(}\PY{o}{\PYZhy{}}\PY{l+m+mf}{0.01}\PY{p}{,} \PY{l+m+mf}{0.65}\PY{p}{)}
        \PY{n}{plt}\PY{o}{.}\PY{n}{show}\PY{p}{(}\PY{p}{)}\PY{p}{;}
\end{Verbatim}


    \begin{center}
    \adjustimage{max size={0.9\linewidth}{0.9\paperheight}}{output_7_0.png}
    \end{center}
    { \hspace*{\fill} \\}
    
    \begin{Verbatim}[commandchars=\\\{\}]
{\color{incolor}In [{\color{incolor}4}]:} \PY{n}{sns}\PY{o}{.}\PY{n}{set}\PY{p}{(}\PY{n}{rc}\PY{o}{=}\PY{p}{\PYZob{}}\PY{l+s+s2}{\PYZdq{}}\PY{l+s+s2}{figure.figsize}\PY{l+s+s2}{\PYZdq{}}\PY{p}{:} \PY{p}{(}\PY{l+m+mi}{12}\PY{p}{,} \PY{l+m+mi}{8}\PY{p}{)}\PY{p}{\PYZcb{}}\PY{p}{)}
        \PY{n}{plt}\PY{o}{.}\PY{n}{style}\PY{o}{.}\PY{n}{use}\PY{p}{(}\PY{l+s+s1}{\PYZsq{}}\PY{l+s+s1}{fivethirtyeight}\PY{l+s+s1}{\PYZsq{}}\PY{p}{)}
        
        \PY{n}{stats}\PY{o}{.}\PY{n}{probplot}\PY{p}{(}\PY{n}{temp}\PY{p}{,} \PY{n}{dist}\PY{o}{=}\PY{l+s+s1}{\PYZsq{}}\PY{l+s+s1}{norm}\PY{l+s+s1}{\PYZsq{}}\PY{p}{,} \PY{n}{plot}\PY{o}{=}\PY{n}{pylab}\PY{p}{)}
        \PY{n}{plt}\PY{o}{.}\PY{n}{title}\PY{p}{(}\PY{l+s+s1}{\PYZsq{}}\PY{l+s+s1}{Fig. 1.2: Probability Plot of Human Body Temperatures}\PY{l+s+s1}{\PYZsq{}}\PY{p}{)}\PY{p}{;}
\end{Verbatim}


    \begin{center}
    \adjustimage{max size={0.9\linewidth}{0.9\paperheight}}{output_8_0.png}
    \end{center}
    { \hspace*{\fill} \\}
    
    \begin{Verbatim}[commandchars=\\\{\}]
{\color{incolor}In [{\color{incolor}5}]:} \PY{n}{sns}\PY{o}{.}\PY{n}{set}\PY{p}{(}\PY{n}{rc}\PY{o}{=}\PY{p}{\PYZob{}}\PY{l+s+s2}{\PYZdq{}}\PY{l+s+s2}{figure.figsize}\PY{l+s+s2}{\PYZdq{}}\PY{p}{:} \PY{p}{(}\PY{l+m+mi}{12}\PY{p}{,} \PY{l+m+mi}{8}\PY{p}{)}\PY{p}{\PYZcb{}}\PY{p}{)}
        \PY{n}{plt}\PY{o}{.}\PY{n}{style}\PY{o}{.}\PY{n}{use}\PY{p}{(}\PY{l+s+s1}{\PYZsq{}}\PY{l+s+s1}{fivethirtyeight}\PY{l+s+s1}{\PYZsq{}}\PY{p}{)}
        
        \PY{c+c1}{\PYZsh{} Plot the CDFs}
        \PY{n}{x}\PY{p}{,} \PY{n}{y} \PY{o}{=} \PY{n}{ecdf}\PY{p}{(}\PY{n}{temp}\PY{p}{)}
        
        \PY{c+c1}{\PYZsh{} draw 100,000 random samples from a normal distribution}
        \PY{n}{nm\PYZus{}temp} \PY{o}{=} \PY{n}{np}\PY{o}{.}\PY{n}{random}\PY{o}{.}\PY{n}{normal}\PY{p}{(}\PY{n}{np}\PY{o}{.}\PY{n}{mean}\PY{p}{(}\PY{n}{temp}\PY{p}{)}\PY{p}{,} \PY{n}{np}\PY{o}{.}\PY{n}{std}\PY{p}{(}\PY{n}{temp}\PY{p}{)}\PY{p}{,} \PY{l+m+mi}{100000}\PY{p}{)}
        \PY{n}{nm\PYZus{}x}\PY{p}{,} \PY{n}{nm\PYZus{}y} \PY{o}{=} \PY{n}{ecdf}\PY{p}{(}\PY{n}{nm\PYZus{}temp}\PY{p}{)}
        
        \PY{n}{\PYZus{}} \PY{o}{=} \PY{n}{plt}\PY{o}{.}\PY{n}{plot}\PY{p}{(}\PY{n}{x}\PY{p}{,} \PY{n}{y}\PY{p}{,} \PY{n}{marker}\PY{o}{=}\PY{l+s+s1}{\PYZsq{}}\PY{l+s+s1}{.}\PY{l+s+s1}{\PYZsq{}}\PY{p}{,} \PY{n}{linestyle}\PY{o}{=}\PY{l+s+s1}{\PYZsq{}}\PY{l+s+s1}{none}\PY{l+s+s1}{\PYZsq{}}\PY{p}{)}
        \PY{n}{\PYZus{}} \PY{o}{=} \PY{n}{plt}\PY{o}{.}\PY{n}{plot}\PY{p}{(}\PY{n}{nm\PYZus{}x}\PY{p}{,} \PY{n}{nm\PYZus{}y}\PY{p}{)}
        
        \PY{n}{\PYZus{}} \PY{o}{=} \PY{n}{plt}\PY{o}{.}\PY{n}{xlabel}\PY{p}{(}\PY{l+s+s1}{\PYZsq{}}\PY{l+s+s1}{Temperature (F)}\PY{l+s+s1}{\PYZsq{}}\PY{p}{)}
        \PY{n}{\PYZus{}} \PY{o}{=} \PY{n}{plt}\PY{o}{.}\PY{n}{ylabel}\PY{p}{(}\PY{l+s+s1}{\PYZsq{}}\PY{l+s+s1}{CDF}\PY{l+s+s1}{\PYZsq{}}\PY{p}{)}
        \PY{n}{\PYZus{}} \PY{o}{=} \PY{n}{plt}\PY{o}{.}\PY{n}{title}\PY{p}{(}\PY{l+s+s1}{\PYZsq{}}\PY{l+s+s1}{Fig. 1.3: Cumulative Distribution Function \PYZhy{} Human Body Temperatures}\PY{l+s+s1}{\PYZsq{}}\PY{p}{)}
        \PY{n}{margins} \PY{o}{=} \PY{l+m+mf}{0.02}
        
        \PY{n}{plt}\PY{o}{.}\PY{n}{show}\PY{p}{(}\PY{p}{)}\PY{p}{;}
\end{Verbatim}


    \begin{center}
    \adjustimage{max size={0.9\linewidth}{0.9\paperheight}}{output_9_0.png}
    \end{center}
    { \hspace*{\fill} \\}
    
    Analysis:

The histogram (Fig. 1.1) shows that the sample's distribution is
unimodal and essentially symmetrical about the mean. The QQ plot
(Fig.1.2) and the CDF (Fig. 1.3), also show that the frequency
distribution of the data is very close to normal, although the QQ plot
and the CDF both show some variation and a couple of outliers on both
tails.

    \subsubsection{Question \#2}\label{question-2}

\begin{center}\rule{0.5\linewidth}{\linethickness}\end{center}

Is the sample size large? Are the observations independent? - Remember
that this is a condition for the Central Limit Theorem, and hence the
statistical tests we are using, to apply.

    \begin{Verbatim}[commandchars=\\\{\}]
{\color{incolor}In [{\color{incolor}6}]:} \PY{n}{df}\PY{o}{.}\PY{n}{info}\PY{p}{(}\PY{p}{)}
\end{Verbatim}


    \begin{Verbatim}[commandchars=\\\{\}]
<class 'pandas.core.frame.DataFrame'>
RangeIndex: 130 entries, 0 to 129
Data columns (total 3 columns):
temperature    130 non-null float64
gender         130 non-null object
heart\_rate     130 non-null float64
dtypes: float64(2), object(1)
memory usage: 3.1+ KB

    \end{Verbatim}

    Analysis:

The sample size is not large as a fraction of all human beings, but it
is large enough for the Central Limit Theorem to apply.

The central limit theorem states that the sampling distribution of the
sample mean approximates the normal distribution, regardless of the
distribution of the population from which the samples are drawn if the
sample size is sufficiently large. This fact enables us to make
statistical inferences based on the properties of the normal
distribution, even if the sample is drawn from a population that is not
normally distributed.

The definition of the central limit theorem reads that the sample size
must be 'sufficiently large', but fails to define 'sufficiently large'.
A commonly-accepted rule of thumb is that a sample size of 30
constitutes 'sufficiently large'. However, if the population from which
the samples are drawn is unimodal and symmetric about the mean, a sample
size of less than 30 may be sufficient. Conversely, if the population is
multi-modal or skewed, a sample size of more than 30 would be required.

The 130 observations is a tiny fraction of the population of all humans,
but, as shown in the analysis of Question \#1, the data is unimodal and
essentially symmetric about the mean, so, a sample size of 130 is more
than sufficient to satisfy the 'sufficiently large' requirement of the
central limit theorem.

For the observations to be independent it would mean that knowing the
outcome of one sample would provide no information about another sample.
In this case, knowing one person's body temperature gives no information
about any other measured body temperature. Gender or heart rate may have
an affect on body temperature. The effects of gender on mean body
temperature are explored below.

    \subsubsection{Question \#3}\label{question-3}

\begin{center}\rule{0.5\linewidth}{\linethickness}\end{center}

Is the true population mean really 98.6 degrees F?

\begin{verbatim}
<li> First, try a bootstrap hypothesis test.
<li> Now, let's try frequentist statistical testing. Would you use a one-sample or two-sample test? Why?
<li> In this situation, is it appropriate to use the $t$ or $z$ statistic? 
<li> Now try using the other test. How is the result be different? Why?
\end{verbatim}

    Hypotheses:

\(H_0:\  \bar x =\) 98.6°\\
\(H_a:\  \bar x \neq\) 98.6°

    \begin{Verbatim}[commandchars=\\\{\}]
{\color{incolor}In [{\color{incolor}7}]:} \PY{n}{df}\PY{o}{.}\PY{n}{describe}\PY{p}{(}\PY{p}{)}
\end{Verbatim}


\begin{Verbatim}[commandchars=\\\{\}]
{\color{outcolor}Out[{\color{outcolor}7}]:}        temperature  heart\_rate
        count   130.000000  130.000000
        mean     98.249231   73.761538
        std       0.733183    7.062077
        min      96.300000   57.000000
        25\%      97.800000   69.000000
        50\%      98.300000   74.000000
        75\%      98.700000   79.000000
        max     100.800000   89.000000
\end{Verbatim}
            
    \paragraph{3.1 bootstrap hypothesis test - 100,000
samples}\label{bootstrap-hypothesis-test---100000-samples}

    \begin{Verbatim}[commandchars=\\\{\}]
{\color{incolor}In [{\color{incolor}8}]:} \PY{c+c1}{\PYZsh{} bootstrap hypothesis test with 100,000 samples}
        \PY{n}{bs\PYZus{}replicates} \PY{o}{=} \PY{n}{np}\PY{o}{.}\PY{n}{empty}\PY{p}{(}\PY{l+m+mi}{100000}\PY{p}{)}
        
        \PY{n}{size} \PY{o}{=} \PY{n+nb}{len}\PY{p}{(}\PY{n}{bs\PYZus{}replicates}\PY{p}{)}
        
        \PY{k}{for} \PY{n}{i} \PY{o+ow}{in} \PY{n+nb}{range}\PY{p}{(}\PY{n}{size}\PY{p}{)}\PY{p}{:}
            \PY{n}{bs\PYZus{}sample} \PY{o}{=} \PY{n}{np}\PY{o}{.}\PY{n}{random}\PY{o}{.}\PY{n}{choice}\PY{p}{(}\PY{n}{temp}\PY{p}{,} \PY{n+nb}{len}\PY{p}{(}\PY{n}{temp}\PY{p}{)}\PY{p}{)}
            \PY{n}{bs\PYZus{}replicates}\PY{p}{[}\PY{n}{i}\PY{p}{]} \PY{o}{=} \PY{n}{np}\PY{o}{.}\PY{n}{mean}\PY{p}{(}\PY{n}{bs\PYZus{}sample}\PY{p}{)}
            
        \PY{n}{p} \PY{o}{=} \PY{n}{np}\PY{o}{.}\PY{n}{sum}\PY{p}{(}\PY{n}{bs\PYZus{}replicates} \PY{o}{\PYZgt{}}\PY{o}{=} \PY{l+m+mf}{98.6}\PY{p}{)} \PY{o}{/} \PY{n}{size}
        
        \PY{n+nb}{print}\PY{p}{(}\PY{l+s+s1}{\PYZsq{}}\PY{l+s+s1}{p\PYZhy{}value: }\PY{l+s+s1}{\PYZsq{}}\PY{p}{,} \PY{n}{p}\PY{p}{)}
        \PY{n+nb}{print}\PY{p}{(}\PY{l+s+s1}{\PYZsq{}}\PY{l+s+s1}{mean: }\PY{l+s+s1}{\PYZsq{}}\PY{p}{,} \PY{n}{np}\PY{o}{.}\PY{n}{mean}\PY{p}{(}\PY{n}{bs\PYZus{}replicates}\PY{p}{)}\PY{p}{)}
\end{Verbatim}


    \begin{Verbatim}[commandchars=\\\{\}]
p-value:  0.0
mean:  98.24910042307691

    \end{Verbatim}

    Analysis:

After 100,000 samples, the p-value is 0.0, indicating that the null
hypothesis should be rejected. The mean body temperature of the sample
set is 98.25°.

    \paragraph{3.2 frequentist statistical
testing}\label{frequentist-statistical-testing}

Now, let's try frequentist statistical testing. Would you use a
one-sample or two-sample test? Why?

    Analysis:

When comparing the mean of a single sample to a population with an
hypothesised mean, a one-sample \(t\)-test is appropriate.

    \paragraph{\texorpdfstring{3.3 / 3.4 \(t\)- or \(Z\)-
statistic?}{3.3 / 3.4 t- or Z- statistic?}}\label{t--or-z--statistic}

    In this situation, is it appropriate to use the \(t\)- or
\(Z\)-statistic?

    Analysis:

\[Z-statistic = Z = \frac{\bar x - \mu}{\frac{\sigma}{\sqrt n}}\]

\[t-statistic = t = \frac{\bar x - \mu}{\frac{s}{\sqrt n}}\]

The formula for the \(Z\)-statistic requires the population standard
deviation, which is unknown. The \(t\)-statistic requires only the
sample standard deviation, which can be derived. Without knowing the
population standard deviation, the only choice is to use the
\(t\)-statistic.

A two-tailed test is required, since the alternative hypothesis is that
there is a statistically-significant difference between the sample mean
and the hypothetical population mean of 98.6°, rather than testing
whether the actual mean temperature is greater than or less than the
hypothesized 98.6°.

    \begin{Verbatim}[commandchars=\\\{\}]
{\color{incolor}In [{\color{incolor}9}]:} \PY{n}{t\PYZus{}stat} \PY{o}{=} \PY{n}{stats}\PY{o}{.}\PY{n}{ttest\PYZus{}1samp}\PY{p}{(}\PY{n}{temp}\PY{p}{,} \PY{l+m+mf}{98.6}\PY{p}{)}
        \PY{n+nb}{print}\PY{p}{(}\PY{l+s+s1}{\PYZsq{}}\PY{l+s+s1}{t\PYZhy{}score: }\PY{l+s+si}{\PYZob{}\PYZcb{}}\PY{l+s+se}{\PYZbs{}n}\PY{l+s+s1}{p\PYZhy{}value: }\PY{l+s+si}{\PYZob{}\PYZcb{}}\PY{l+s+s1}{\PYZsq{}}\PY{o}{.}\PY{n}{format}\PY{p}{(}\PY{n+nb}{round}\PY{p}{(}\PY{n}{t\PYZus{}stat}\PY{o}{.}\PY{n}{statistic}\PY{p}{,} \PY{l+m+mi}{5}\PY{p}{)}\PY{p}{,} \PY{n+nb}{round}\PY{p}{(}\PY{n}{t\PYZus{}stat}\PY{o}{.}\PY{n}{pvalue}\PY{p}{,} \PY{l+m+mi}{5}\PY{p}{)}\PY{p}{)}\PY{p}{)}
\end{Verbatim}


    \begin{Verbatim}[commandchars=\\\{\}]
t-score: -5.45482
p-value: 0.0

    \end{Verbatim}

    \subsubsection{Question \#4}\label{question-4}

\begin{center}\rule{0.5\linewidth}{\linethickness}\end{center}

Draw a small sample of size 10 from the data and repeat both frequentist
tests.

\begin{verbatim}
<li> Which one is the correct one to use? 
<li> What do you notice? What does this tell you about the difference in application of the $t$ and $z$ statistic?
</ul>
\end{verbatim}

    \paragraph{\texorpdfstring{\(t\)-test}{t-test}}\label{t-test}

    \begin{Verbatim}[commandchars=\\\{\}]
{\color{incolor}In [{\color{incolor}10}]:} \PY{n}{sample\PYZus{}temp} \PY{o}{=} \PY{n}{np}\PY{o}{.}\PY{n}{random}\PY{o}{.}\PY{n}{choice}\PY{p}{(}\PY{n}{a} \PY{o}{=} \PY{n}{temp}\PY{p}{,} \PY{n}{size}\PY{o}{=}\PY{l+m+mi}{10}\PY{p}{)}
\end{Verbatim}


    \begin{Verbatim}[commandchars=\\\{\}]
{\color{incolor}In [{\color{incolor}11}]:} \PY{n}{r} \PY{o}{=} \PY{n}{stats}\PY{o}{.}\PY{n}{ttest\PYZus{}1samp}\PY{p}{(}\PY{n}{sample\PYZus{}temp}\PY{p}{,} \PY{l+m+mf}{98.6}\PY{p}{)}
         \PY{n+nb}{print}\PY{p}{(}\PY{l+s+s1}{\PYZsq{}}\PY{l+s+s1}{t\PYZhy{}score: }\PY{l+s+si}{\PYZob{}:0.4\PYZcb{}}\PY{l+s+se}{\PYZbs{}n}\PY{l+s+s1}{p\PYZhy{}value: }\PY{l+s+si}{\PYZob{}:0.4\PYZcb{}}\PY{l+s+s1}{\PYZsq{}}\PY{o}{.}\PY{n}{format}\PY{p}{(}\PY{n}{r}\PY{o}{.}\PY{n}{statistic}\PY{p}{,} \PY{n}{r}\PY{o}{.}\PY{n}{pvalue}\PY{p}{)}\PY{p}{)}
\end{Verbatim}


    \begin{Verbatim}[commandchars=\\\{\}]
t-score: -1.546
p-value: 0.1565

    \end{Verbatim}

    Analysis:

Again, since the population standard deviation is unknown, the
\(t\)-test is the only option available. The p-value is greater than
0.05, so the null hypothesis cannot be rejected on the basis of this
test.

    \subsubsection{Question \#5}\label{question-5}

\begin{center}\rule{0.5\linewidth}{\linethickness}\end{center}

At what temperature should we consider someone's temperature to be
"abnormal"?

\begin{verbatim}
<li> As in the previous example, try calculating everything using the boostrap approach, as well as the frequentist approach.
<li> Start by computing the margin of error and confidence interval. When calculating the confidence interval, keep in mind that you should use the appropriate formula for one draw, and not N draws.
\end{verbatim}

    \begin{Verbatim}[commandchars=\\\{\}]
{\color{incolor}In [{\color{incolor}12}]:} \PY{c+c1}{\PYZsh{} get the sample mean and standard deviation for use with bootstrap and frequentist approaches below}
         \PY{n}{x\PYZus{}bar} \PY{o}{=} \PY{n}{np}\PY{o}{.}\PY{n}{mean}\PY{p}{(}\PY{n}{temp}\PY{p}{)}
         \PY{n}{s} \PY{o}{=} \PY{n}{np}\PY{o}{.}\PY{n}{std}\PY{p}{(}\PY{n}{temp}\PY{p}{)}
         \PY{n+nb}{print}\PY{p}{(}\PY{l+s+s1}{\PYZsq{}}\PY{l+s+s1}{Sample mean: }\PY{l+s+si}{\PYZob{}:0.4\PYZcb{}}\PY{l+s+se}{\PYZbs{}n}\PY{l+s+s1}{Sample Standard Deviation: }\PY{l+s+si}{\PYZob{}:0.4\PYZcb{}}\PY{l+s+se}{\PYZbs{}n}\PY{l+s+s1}{\PYZsq{}}\PY{o}{.}\PY{n}{format}\PY{p}{(}\PY{n}{x\PYZus{}bar}\PY{p}{,} \PY{n}{s}\PY{p}{)}\PY{p}{)}
\end{Verbatim}


    \begin{Verbatim}[commandchars=\\\{\}]
Sample mean: 98.25
Sample Standard Deviation: 0.7304


    \end{Verbatim}

    \paragraph{Bootstrap}\label{bootstrap}

    \begin{Verbatim}[commandchars=\\\{\}]
{\color{incolor}In [{\color{incolor}13}]:} \PY{c+c1}{\PYZsh{} bootstrap with 10,000 samples}
         \PY{n}{size} \PY{o}{=} \PY{l+m+mi}{10000}
         \PY{n}{bs\PYZus{}ci\PYZus{}low} \PY{o}{=} \PY{n}{np}\PY{o}{.}\PY{n}{empty}\PY{p}{(}\PY{n}{size}\PY{p}{)}
         \PY{n}{bs\PYZus{}ci\PYZus{}high} \PY{o}{=} \PY{n}{np}\PY{o}{.}\PY{n}{empty}\PY{p}{(}\PY{n}{size}\PY{p}{)}
         
         
         \PY{k}{for} \PY{n}{i} \PY{o+ow}{in} \PY{n+nb}{range}\PY{p}{(}\PY{n}{size}\PY{p}{)}\PY{p}{:}
             \PY{n}{bs\PYZus{}sample} \PY{o}{=} \PY{n}{np}\PY{o}{.}\PY{n}{random}\PY{o}{.}\PY{n}{choice}\PY{p}{(}\PY{n}{temp}\PY{p}{,} \PY{n+nb}{len}\PY{p}{(}\PY{n}{temp}\PY{p}{)}\PY{p}{)}
             \PY{n}{bs\PYZus{}replicate} \PY{o}{=} \PY{n}{stats}\PY{o}{.}\PY{n}{norm}\PY{o}{.}\PY{n}{interval}\PY{p}{(}\PY{l+m+mf}{0.95}\PY{p}{,} \PY{n}{loc}\PY{o}{=}\PY{n}{x\PYZus{}bar}\PY{p}{,} \PY{n}{scale}\PY{o}{=}\PY{n}{s}\PY{p}{)}
             \PY{n}{bs\PYZus{}ci\PYZus{}low}\PY{p}{[}\PY{n}{i}\PY{p}{]} \PY{o}{=} \PY{n}{bs\PYZus{}replicate}\PY{p}{[}\PY{l+m+mi}{0}\PY{p}{]}
             \PY{n}{bs\PYZus{}ci\PYZus{}high}\PY{p}{[}\PY{n}{i}\PY{p}{]} \PY{o}{=} \PY{n}{bs\PYZus{}replicate}\PY{p}{[}\PY{l+m+mi}{1}\PY{p}{]}
             
         \PY{n}{ci\PYZus{}low\PYZus{}b} \PY{o}{=} \PY{n}{np}\PY{o}{.}\PY{n}{sum}\PY{p}{(}\PY{n}{bs\PYZus{}ci\PYZus{}low}\PY{p}{)} \PY{o}{/} \PY{n}{size}
         \PY{n}{ci\PYZus{}high\PYZus{}b} \PY{o}{=} \PY{n}{np}\PY{o}{.}\PY{n}{sum}\PY{p}{(}\PY{n}{bs\PYZus{}ci\PYZus{}high}\PY{p}{)} \PY{o}{/} \PY{n}{size}
         
         \PY{n+nb}{print}\PY{p}{(}\PY{l+s+s1}{\PYZsq{}}\PY{l+s+s1}{95}\PY{l+s+si}{\PYZpc{} c}\PY{l+s+s1}{onfidence interval}\PY{l+s+se}{\PYZbs{}n}\PY{l+s+si}{\PYZob{}:0.5\PYZcb{}}\PY{l+s+s1}{ \PYZhy{} }\PY{l+s+si}{\PYZob{}:0.5\PYZcb{}}\PY{l+s+se}{\PYZbs{}n}\PY{l+s+se}{\PYZbs{}n}\PY{l+s+s1}{\PYZsq{}}\PY{o}{.}\PY{n}{format}\PY{p}{(}\PY{n}{ci\PYZus{}low\PYZus{}b}\PY{p}{,} \PY{n}{ci\PYZus{}high\PYZus{}b}\PY{p}{)}\PY{p}{)}
         \PY{n+nb}{print}\PY{p}{(}\PY{l+s+s1}{\PYZsq{}}\PY{l+s+s1}{margin of error}\PY{l+s+se}{\PYZbs{}n}\PY{l+s+si}{\PYZob{}:0.5\PYZcb{}}\PY{l+s+s1}{ \PYZhy{} }\PY{l+s+si}{\PYZob{}:0.5\PYZcb{}}\PY{l+s+s1}{\PYZsq{}}\PY{o}{.}\PY{n}{format}\PY{p}{(}\PY{p}{(}\PY{n}{ci\PYZus{}low\PYZus{}b} \PY{o}{\PYZhy{}} \PY{n}{x\PYZus{}bar}\PY{p}{)}\PY{p}{,} \PY{p}{(}\PY{n}{ci\PYZus{}high\PYZus{}b} \PY{o}{\PYZhy{}} \PY{n}{x\PYZus{}bar}\PY{p}{)}\PY{p}{)}\PY{p}{)}
\end{Verbatim}


    \begin{Verbatim}[commandchars=\\\{\}]
95\% confidence interval
96.818 - 99.681


margin of error
-1.4315 - 1.4315

    \end{Verbatim}

    \paragraph{Frequentist Approach}\label{frequentist-approach}

Confidence Interval for One-Sample t-Test
\[CI_{1-\alpha} = \bar x \pm \left(t_{\frac{\alpha}{2}df}\right)\left(\frac{s}{\sqrt n}\right)\]

    \begin{Verbatim}[commandchars=\\\{\}]
{\color{incolor}In [{\color{incolor}14}]:} \PY{c+c1}{\PYZsh{} frequentist approach \PYZhy{} confidence interval for the one\PYZhy{}sample t\PYZhy{}test}
         
         \PY{c+c1}{\PYZsh{} alpha = 0.05, confidence coefficient = 95\PYZpc{}}
         
         \PY{c+c1}{\PYZsh{} confidence interval for one draw}
         \PY{n}{ci\PYZus{}low\PYZus{}f}\PY{p}{,} \PY{n}{ci\PYZus{}high\PYZus{}f} \PY{o}{=} \PY{n}{stats}\PY{o}{.}\PY{n}{norm}\PY{o}{.}\PY{n}{interval}\PY{p}{(}\PY{l+m+mf}{0.95}\PY{p}{,} \PY{n}{loc}\PY{o}{=}\PY{n}{x\PYZus{}bar}\PY{p}{,} \PY{n}{scale}\PY{o}{=}\PY{n}{s}\PY{p}{)}
         \PY{n+nb}{print}\PY{p}{(}\PY{l+s+s1}{\PYZsq{}}\PY{l+s+s1}{95}\PY{l+s+si}{\PYZpc{} c}\PY{l+s+s1}{onfidence interval (one draw)}\PY{l+s+se}{\PYZbs{}n}\PY{l+s+si}{\PYZob{}:0.5\PYZcb{}}\PY{l+s+s1}{ \PYZhy{} }\PY{l+s+si}{\PYZob{}:0.5\PYZcb{}}\PY{l+s+se}{\PYZbs{}n}\PY{l+s+se}{\PYZbs{}n}\PY{l+s+s1}{\PYZsq{}}\PY{o}{.}\PY{n}{format}\PY{p}{(}\PY{n}{ci\PYZus{}low\PYZus{}f}\PY{p}{,} \PY{n}{ci\PYZus{}high\PYZus{}f}\PY{p}{)}\PY{p}{)}
         
         
         \PY{c+c1}{\PYZsh{} margin of error }
         \PY{n+nb}{print}\PY{p}{(}\PY{l+s+s1}{\PYZsq{}}\PY{l+s+s1}{margin of error}\PY{l+s+se}{\PYZbs{}n}\PY{l+s+si}{\PYZob{}:0.5\PYZcb{}}\PY{l+s+s1}{ \PYZhy{} }\PY{l+s+si}{\PYZob{}:0.5\PYZcb{}}\PY{l+s+s1}{\PYZsq{}}\PY{o}{.}\PY{n}{format}\PY{p}{(}\PY{p}{(}\PY{n}{ci\PYZus{}low\PYZus{}f} \PY{o}{\PYZhy{}} \PY{n}{x\PYZus{}bar}\PY{p}{)}\PY{p}{,} \PY{p}{(}\PY{n}{ci\PYZus{}high\PYZus{}f} \PY{o}{\PYZhy{}} \PY{n}{x\PYZus{}bar}\PY{p}{)}\PY{p}{)}\PY{p}{)} 
\end{Verbatim}


    \begin{Verbatim}[commandchars=\\\{\}]
95\% confidence interval (one draw)
96.818 - 99.681


margin of error
-1.4315 - 1.4315

    \end{Verbatim}

    Analysis:

According to both the Bootstrap and Frequentist approaches, using the
mean we calculated (98.249°), and at a 95\% confidence interval, a
temperature below 96.818° or above 99.681° would be considered abnormal.

    \subsubsection{Question \#6}\label{question-6}

\begin{center}\rule{0.5\linewidth}{\linethickness}\end{center}

Is there a significant difference between males and females in normal
temperature?

\begin{verbatim}
<li> What testing approach did you use and why?
<li> Write a story with your conclusion in the context of the original problem.
\end{verbatim}

    \subsubsection{Hypotheses}\label{hypotheses}

\(H_0: \bar x_m\ =\ \bar x_f\)

\(H_a: \bar x_m \neq\ \bar x_f\)

    \begin{Verbatim}[commandchars=\\\{\}]
{\color{incolor}In [{\color{incolor}15}]:} \PY{n}{males} \PY{o}{=} \PY{n}{df}\PY{p}{[}\PY{n}{df}\PY{o}{.}\PY{n}{gender} \PY{o}{==} \PY{l+s+s1}{\PYZsq{}}\PY{l+s+s1}{M}\PY{l+s+s1}{\PYZsq{}}\PY{p}{]}
         \PY{n}{females} \PY{o}{=} \PY{n}{df}\PY{p}{[}\PY{n}{df}\PY{o}{.}\PY{n}{gender} \PY{o}{==} \PY{l+s+s1}{\PYZsq{}}\PY{l+s+s1}{F}\PY{l+s+s1}{\PYZsq{}}\PY{p}{]}
         
         \PY{n+nb}{print}\PY{p}{(}\PY{l+s+s1}{\PYZsq{}}\PY{l+s+s1}{Of the }\PY{l+s+si}{\PYZob{}\PYZcb{}}\PY{l+s+s1}{ participants, }\PY{l+s+si}{\PYZob{}\PYZcb{}}\PY{l+s+s1}{ are female and }\PY{l+s+si}{\PYZob{}\PYZcb{}}\PY{l+s+s1}{ are male.}\PY{l+s+s1}{\PYZsq{}}\PY{o}{.}\PY{n}{format}\PY{p}{(}\PY{n+nb}{len}\PY{p}{(}\PY{n}{males} \PY{o}{+} \PY{n}{females}\PY{p}{)}\PY{p}{,} \PY{n+nb}{len}\PY{p}{(}\PY{n}{females}\PY{p}{)}\PY{p}{,} \PY{n+nb}{len}\PY{p}{(}\PY{n}{males}\PY{p}{)}\PY{p}{)}\PY{p}{)}
\end{Verbatim}


    \begin{Verbatim}[commandchars=\\\{\}]
Of the 130 participants, 65 are female and 65 are male.

    \end{Verbatim}

    \begin{Verbatim}[commandchars=\\\{\}]
{\color{incolor}In [{\color{incolor}16}]:} \PY{c+c1}{\PYZsh{} plot a boxplot for an overview}
         \PY{n}{sns}\PY{o}{.}\PY{n}{boxplot}\PY{p}{(}\PY{n}{x} \PY{o}{=} \PY{l+s+s1}{\PYZsq{}}\PY{l+s+s1}{gender}\PY{l+s+s1}{\PYZsq{}}\PY{p}{,} \PY{n}{y} \PY{o}{=} \PY{l+s+s1}{\PYZsq{}}\PY{l+s+s1}{temperature}\PY{l+s+s1}{\PYZsq{}}\PY{p}{,} \PY{n}{data}\PY{o}{=}\PY{n}{df}\PY{p}{)}
         
         \PY{n}{sns}\PY{o}{.}\PY{n}{set}\PY{p}{(}\PY{n}{rc}\PY{o}{=}\PY{p}{\PYZob{}}\PY{l+s+s2}{\PYZdq{}}\PY{l+s+s2}{figure.figsize}\PY{l+s+s2}{\PYZdq{}}\PY{p}{:} \PY{p}{(}\PY{l+m+mi}{12}\PY{p}{,} \PY{l+m+mi}{8}\PY{p}{)}\PY{p}{\PYZcb{}}\PY{p}{)}
         \PY{n}{plt}\PY{o}{.}\PY{n}{style}\PY{o}{.}\PY{n}{use}\PY{p}{(}\PY{l+s+s1}{\PYZsq{}}\PY{l+s+s1}{fivethirtyeight}\PY{l+s+s1}{\PYZsq{}}\PY{p}{)}
         
         \PY{n}{\PYZus{}} \PY{o}{=} \PY{n}{plt}\PY{o}{.}\PY{n}{xlabel}\PY{p}{(}\PY{l+s+s1}{\PYZsq{}}\PY{l+s+s1}{Gender}\PY{l+s+s1}{\PYZsq{}}\PY{p}{)}
         \PY{n}{\PYZus{}} \PY{o}{=} \PY{n}{plt}\PY{o}{.}\PY{n}{ylabel}\PY{p}{(}\PY{l+s+s1}{\PYZsq{}}\PY{l+s+s1}{Temp}\PY{l+s+s1}{\PYZsq{}}\PY{p}{)}
         \PY{n}{\PYZus{}} \PY{o}{=} \PY{n}{plt}\PY{o}{.}\PY{n}{title}\PY{p}{(}\PY{l+s+s1}{\PYZsq{}}\PY{l+s+s1}{Fig. 6.1: Body Temp by Gender}\PY{l+s+s1}{\PYZsq{}}\PY{p}{)}
         
         \PY{n}{plt}\PY{o}{.}\PY{n}{show}\PY{p}{(}\PY{p}{)}\PY{p}{;}
\end{Verbatim}


    \begin{center}
    \adjustimage{max size={0.9\linewidth}{0.9\paperheight}}{output_41_0.png}
    \end{center}
    { \hspace*{\fill} \\}
    
    \begin{Verbatim}[commandchars=\\\{\}]
{\color{incolor}In [{\color{incolor}17}]:} \PY{c+c1}{\PYZsh{} are both samples normally distributed?}
         \PY{n}{sns}\PY{o}{.}\PY{n}{set}\PY{p}{(}\PY{n}{rc}\PY{o}{=}\PY{p}{\PYZob{}}\PY{l+s+s2}{\PYZdq{}}\PY{l+s+s2}{figure.figsize}\PY{l+s+s2}{\PYZdq{}}\PY{p}{:} \PY{p}{(}\PY{l+m+mi}{12}\PY{p}{,} \PY{l+m+mi}{8}\PY{p}{)}\PY{p}{\PYZcb{}}\PY{p}{)}
         \PY{n}{plt}\PY{o}{.}\PY{n}{style}\PY{o}{.}\PY{n}{use}\PY{p}{(}\PY{l+s+s1}{\PYZsq{}}\PY{l+s+s1}{fivethirtyeight}\PY{l+s+s1}{\PYZsq{}}\PY{p}{)}
         
         \PY{c+c1}{\PYZsh{} Compute the ECDFs for males and females}
         \PY{n}{x\PYZus{}male}\PY{p}{,} \PY{n}{y\PYZus{}male} \PY{o}{=} \PY{n}{ecdf}\PY{p}{(}\PY{n}{males}\PY{o}{.}\PY{n}{temperature}\PY{p}{)}
         \PY{n}{x\PYZus{}female}\PY{p}{,} \PY{n}{y\PYZus{}female} \PY{o}{=} \PY{n}{ecdf}\PY{p}{(}\PY{n}{females}\PY{o}{.}\PY{n}{temperature}\PY{p}{)}
         
         \PY{c+c1}{\PYZsh{} Generate plot}
         \PY{n}{plt}\PY{o}{.}\PY{n}{plot}\PY{p}{(}\PY{n}{x\PYZus{}male}\PY{p}{,} \PY{n}{y\PYZus{}male}\PY{p}{,} \PY{n}{marker} \PY{o}{=} \PY{l+s+s1}{\PYZsq{}}\PY{l+s+s1}{.}\PY{l+s+s1}{\PYZsq{}}\PY{p}{,} \PY{n}{linestyle} \PY{o}{=} \PY{l+s+s1}{\PYZsq{}}\PY{l+s+s1}{none}\PY{l+s+s1}{\PYZsq{}}\PY{p}{,} \PY{n}{color}\PY{o}{=}\PY{l+s+s1}{\PYZsq{}}\PY{l+s+s1}{b}\PY{l+s+s1}{\PYZsq{}}\PY{p}{)}
         \PY{n}{plt}\PY{o}{.}\PY{n}{plot}\PY{p}{(}\PY{n}{x\PYZus{}female}\PY{p}{,} \PY{n}{y\PYZus{}female}\PY{p}{,} \PY{n}{marker}\PY{o}{=}\PY{l+s+s1}{\PYZsq{}}\PY{l+s+s1}{.}\PY{l+s+s1}{\PYZsq{}}\PY{p}{,} \PY{n}{linestyle}\PY{o}{=}\PY{l+s+s1}{\PYZsq{}}\PY{l+s+s1}{none}\PY{l+s+s1}{\PYZsq{}}\PY{p}{,} \PY{n}{color}\PY{o}{=}\PY{l+s+s1}{\PYZsq{}}\PY{l+s+s1}{r}\PY{l+s+s1}{\PYZsq{}}\PY{p}{)}
         
         \PY{c+c1}{\PYZsh{} draw 100,000 random samples from a normal distribution}
         \PY{n}{m\PYZus{}norm\PYZus{}dist} \PY{o}{=} \PY{n}{np}\PY{o}{.}\PY{n}{random}\PY{o}{.}\PY{n}{normal}\PY{p}{(}\PY{n}{np}\PY{o}{.}\PY{n}{mean}\PY{p}{(}\PY{n}{males}\PY{o}{.}\PY{n}{temperature}\PY{p}{)}\PY{p}{,} \PY{n}{np}\PY{o}{.}\PY{n}{std}\PY{p}{(}\PY{n}{males}\PY{o}{.}\PY{n}{temperature}\PY{p}{)}\PY{p}{,} \PY{l+m+mi}{100000}\PY{p}{)}
         \PY{n}{mnd\PYZus{}x}\PY{p}{,} \PY{n}{mnd\PYZus{}y} \PY{o}{=} \PY{n}{ecdf}\PY{p}{(}\PY{n}{m\PYZus{}norm\PYZus{}dist}\PY{p}{)}
         \PY{n}{\PYZus{}} \PY{o}{=} \PY{n}{plt}\PY{o}{.}\PY{n}{plot}\PY{p}{(}\PY{n}{mnd\PYZus{}x}\PY{p}{,} \PY{n}{mnd\PYZus{}y}\PY{p}{,} \PY{n}{color}\PY{o}{=}\PY{l+s+s1}{\PYZsq{}}\PY{l+s+s1}{b}\PY{l+s+s1}{\PYZsq{}}\PY{p}{,} \PY{n}{alpha}\PY{o}{=}\PY{l+m+mf}{0.5}\PY{p}{)}
         
         \PY{c+c1}{\PYZsh{} draw 100,000 random samples from a normal distribution}
         \PY{n}{f\PYZus{}norm\PYZus{}dist} \PY{o}{=} \PY{n}{np}\PY{o}{.}\PY{n}{random}\PY{o}{.}\PY{n}{normal}\PY{p}{(}\PY{n}{np}\PY{o}{.}\PY{n}{mean}\PY{p}{(}\PY{n}{females}\PY{o}{.}\PY{n}{temperature}\PY{p}{)}\PY{p}{,} \PY{n}{np}\PY{o}{.}\PY{n}{std}\PY{p}{(}\PY{n}{females}\PY{o}{.}\PY{n}{temperature}\PY{p}{)}\PY{p}{,} \PY{l+m+mi}{100000}\PY{p}{)}
         \PY{n}{fnd\PYZus{}x}\PY{p}{,} \PY{n}{fnd\PYZus{}y} \PY{o}{=} \PY{n}{ecdf}\PY{p}{(}\PY{n}{f\PYZus{}norm\PYZus{}dist}\PY{p}{)}
         \PY{n}{\PYZus{}} \PY{o}{=} \PY{n}{plt}\PY{o}{.}\PY{n}{plot}\PY{p}{(}\PY{n}{fnd\PYZus{}x}\PY{p}{,} \PY{n}{fnd\PYZus{}y}\PY{p}{,} \PY{n}{color}\PY{o}{=}\PY{l+s+s1}{\PYZsq{}}\PY{l+s+s1}{r}\PY{l+s+s1}{\PYZsq{}}\PY{p}{,} \PY{n}{alpha}\PY{o}{=}\PY{l+m+mf}{0.5}\PY{p}{)}
         
         \PY{c+c1}{\PYZsh{} Make the margins nice}
         \PY{n}{plt}\PY{o}{.}\PY{n}{margins} \PY{o}{=} \PY{l+m+mf}{0.02}
         
         \PY{c+c1}{\PYZsh{} Label the axes}
         \PY{n}{\PYZus{}} \PY{o}{=} \PY{n}{plt}\PY{o}{.}\PY{n}{xlabel}\PY{p}{(}\PY{l+s+s1}{\PYZsq{}}\PY{l+s+s1}{Gender}\PY{l+s+s1}{\PYZsq{}}\PY{p}{)}
         \PY{n}{\PYZus{}} \PY{o}{=} \PY{n}{plt}\PY{o}{.}\PY{n}{ylabel}\PY{p}{(}\PY{l+s+s1}{\PYZsq{}}\PY{l+s+s1}{ECDF}\PY{l+s+s1}{\PYZsq{}}\PY{p}{)}
         \PY{n}{\PYZus{}} \PY{o}{=} \PY{n}{plt}\PY{o}{.}\PY{n}{title}\PY{p}{(}\PY{l+s+s1}{\PYZsq{}}\PY{l+s+s1}{Fig. 6.2: CDFs of males v. females}\PY{l+s+s1}{\PYZsq{}}\PY{p}{)}
         \PY{n}{\PYZus{}} \PY{o}{=} \PY{n}{plt}\PY{o}{.}\PY{n}{legend}\PY{p}{(}\PY{p}{(}\PY{l+s+s1}{\PYZsq{}}\PY{l+s+s1}{males}\PY{l+s+s1}{\PYZsq{}}\PY{p}{,} \PY{l+s+s1}{\PYZsq{}}\PY{l+s+s1}{females}\PY{l+s+s1}{\PYZsq{}}\PY{p}{)}\PY{p}{,} \PY{n}{loc}\PY{o}{=}\PY{l+s+s1}{\PYZsq{}}\PY{l+s+s1}{lower right}\PY{l+s+s1}{\PYZsq{}}\PY{p}{,} \PY{n}{fontsize}\PY{o}{=}\PY{l+s+s1}{\PYZsq{}}\PY{l+s+s1}{large}\PY{l+s+s1}{\PYZsq{}}\PY{p}{,} \PY{n}{markerscale}\PY{o}{=}\PY{l+m+mi}{2}\PY{p}{)}
         
         \PY{n}{plt}\PY{o}{.}\PY{n}{show}\PY{p}{(}\PY{p}{)}\PY{p}{;}
\end{Verbatim}


    \begin{center}
    \adjustimage{max size={0.9\linewidth}{0.9\paperheight}}{output_42_0.png}
    \end{center}
    { \hspace*{\fill} \\}
    
    \begin{Verbatim}[commandchars=\\\{\}]
{\color{incolor}In [{\color{incolor}18}]:} \PY{c+c1}{\PYZsh{} common variables}
         \PY{n}{temp\PYZus{}m} \PY{o}{=} \PY{n}{males}\PY{o}{.}\PY{n}{temperature}
         \PY{n}{temp\PYZus{}f} \PY{o}{=} \PY{n}{females}\PY{o}{.}\PY{n}{temperature}
\end{Verbatim}


    \begin{Verbatim}[commandchars=\\\{\}]
{\color{incolor}In [{\color{incolor}19}]:} \PY{c+c1}{\PYZsh{} Check for identical variances}
         \PY{n}{mv} \PY{o}{=} \PY{n}{np}\PY{o}{.}\PY{n}{var}\PY{p}{(}\PY{n}{temp\PYZus{}m}\PY{p}{)}
         \PY{n}{fv} \PY{o}{=} \PY{n}{np}\PY{o}{.}\PY{n}{var}\PY{p}{(}\PY{n}{temp\PYZus{}f}\PY{p}{)}
         
         \PY{n+nb}{print}\PY{p}{(}\PY{l+s+s1}{\PYZsq{}}\PY{l+s+s1}{Male variance: }\PY{l+s+si}{\PYZob{}\PYZcb{}}\PY{l+s+se}{\PYZbs{}n}\PY{l+s+s1}{Female variance: }\PY{l+s+si}{\PYZob{}\PYZcb{}}\PY{l+s+s1}{\PYZsq{}}\PY{o}{.}\PY{n}{format}\PY{p}{(}\PY{n}{mv}\PY{p}{,} \PY{n}{fv}\PY{p}{)}\PY{p}{)}
\end{Verbatim}


    \begin{Verbatim}[commandchars=\\\{\}]
Male variance: 0.4807479289940825
Female variance: 0.5442698224852062

    \end{Verbatim}

    \begin{Verbatim}[commandchars=\\\{\}]
{\color{incolor}In [{\color{incolor}20}]:} \PY{c+c1}{\PYZsh{} confirm that variances are not equal with bootstrap \PYZhy{} null hypothesis is that they are equal}
         
         \PY{n}{size} \PY{o}{=} \PY{l+m+mi}{10000}
         
         \PY{n}{bs\PYZus{}replicates\PYZus{}m} \PY{o}{=} \PY{n}{np}\PY{o}{.}\PY{n}{empty}\PY{p}{(}\PY{n}{size}\PY{p}{)}
         
         \PY{k}{for} \PY{n}{i} \PY{o+ow}{in} \PY{n+nb}{range}\PY{p}{(}\PY{n}{size}\PY{p}{)}\PY{p}{:}
             \PY{n}{bs\PYZus{}sample\PYZus{}m} \PY{o}{=} \PY{n}{np}\PY{o}{.}\PY{n}{random}\PY{o}{.}\PY{n}{choice}\PY{p}{(}\PY{n}{temp\PYZus{}m}\PY{p}{,} \PY{n+nb}{len}\PY{p}{(}\PY{n}{temp\PYZus{}m}\PY{p}{)}\PY{p}{)}
             \PY{n}{bs\PYZus{}replicates\PYZus{}m}\PY{p}{[}\PY{n}{i}\PY{p}{]} \PY{o}{=} \PY{n}{np}\PY{o}{.}\PY{n}{var}\PY{p}{(}\PY{n}{bs\PYZus{}sample\PYZus{}m}\PY{p}{)}
             
         \PY{n}{bs\PYZus{}var\PYZus{}m} \PY{o}{=} \PY{n}{np}\PY{o}{.}\PY{n}{sum}\PY{p}{(}\PY{n}{bs\PYZus{}replicates\PYZus{}m}\PY{p}{)}\PY{o}{/}\PY{n}{size}
         
         \PY{n}{bs\PYZus{}replicates\PYZus{}f} \PY{o}{=} \PY{n}{np}\PY{o}{.}\PY{n}{empty}\PY{p}{(}\PY{n}{size}\PY{p}{)}
         
         \PY{k}{for} \PY{n}{i} \PY{o+ow}{in} \PY{n+nb}{range}\PY{p}{(}\PY{n}{size}\PY{p}{)}\PY{p}{:}
             \PY{n}{bs\PYZus{}sample\PYZus{}f} \PY{o}{=} \PY{n}{np}\PY{o}{.}\PY{n}{random}\PY{o}{.}\PY{n}{choice}\PY{p}{(}\PY{n}{temp\PYZus{}f}\PY{p}{,} \PY{n+nb}{len}\PY{p}{(}\PY{n}{temp\PYZus{}f}\PY{p}{)}\PY{p}{)}
             \PY{n}{bs\PYZus{}replicates\PYZus{}f}\PY{p}{[}\PY{n}{i}\PY{p}{]} \PY{o}{=} \PY{n}{np}\PY{o}{.}\PY{n}{var}\PY{p}{(}\PY{n}{bs\PYZus{}sample\PYZus{}f}\PY{p}{)}
         
         \PY{n}{bs\PYZus{}var\PYZus{}f} \PY{o}{=} \PY{n}{np}\PY{o}{.}\PY{n}{sum}\PY{p}{(}\PY{n}{bs\PYZus{}replicates\PYZus{}f}\PY{p}{)}\PY{o}{/}\PY{n}{size}
         
         \PY{n}{bs\PYZus{}var\PYZus{}m}
         \PY{n}{bs\PYZus{}var\PYZus{}f}
         \PY{n+nb}{print}\PY{p}{(}\PY{l+s+s1}{\PYZsq{}}\PY{l+s+s1}{Bootstrap verification:}\PY{l+s+se}{\PYZbs{}n}\PY{l+s+s1}{Male variance: }\PY{l+s+si}{\PYZob{}\PYZcb{}}\PY{l+s+se}{\PYZbs{}n}\PY{l+s+s1}{Female variance: }\PY{l+s+si}{\PYZob{}\PYZcb{}}\PY{l+s+s1}{\PYZsq{}}\PY{o}{.}\PY{n}{format}\PY{p}{(}\PY{n}{bs\PYZus{}var\PYZus{}m}\PY{p}{,} \PY{n}{bs\PYZus{}var\PYZus{}f}\PY{p}{)}\PY{p}{)}
\end{Verbatim}


\begin{Verbatim}[commandchars=\\\{\}]
{\color{outcolor}Out[{\color{outcolor}20}]:} 0.4739684009467454
\end{Verbatim}
            
\begin{Verbatim}[commandchars=\\\{\}]
{\color{outcolor}Out[{\color{outcolor}20}]:} 0.5372722897041412
\end{Verbatim}
            
    \begin{Verbatim}[commandchars=\\\{\}]
Bootstrap verification:
Male variance: 0.4739684009467454
Female variance: 0.5372722897041412

    \end{Verbatim}

    \begin{Verbatim}[commandchars=\\\{\}]
{\color{incolor}In [{\color{incolor}21}]:} \PY{c+c1}{\PYZsh{} Variances are not identical, so set `equal\PYZus{}var` to false to perform Welch\PYZsq{}s t\PYZhy{}test}
         \PY{n}{r} \PY{o}{=} \PY{n}{stats}\PY{o}{.}\PY{n}{ttest\PYZus{}ind}\PY{p}{(}\PY{n}{temp\PYZus{}m}\PY{p}{,} \PY{n}{temp\PYZus{}f}\PY{p}{,} \PY{n}{equal\PYZus{}var}\PY{o}{=}\PY{k+kc}{False}\PY{p}{)}
         \PY{n+nb}{print}\PY{p}{(}\PY{l+s+s1}{\PYZsq{}}\PY{l+s+s1}{t\PYZhy{}statistic: }\PY{l+s+si}{\PYZob{}:0.4\PYZcb{}}\PY{l+s+se}{\PYZbs{}n}\PY{l+s+s1}{p\PYZhy{}value: }\PY{l+s+si}{\PYZob{}:0.4\PYZcb{}}\PY{l+s+s1}{\PYZsq{}}\PY{o}{.}\PY{n}{format}\PY{p}{(}\PY{n}{r}\PY{o}{.}\PY{n}{statistic}\PY{p}{,} \PY{n}{r}\PY{o}{.}\PY{n}{pvalue}\PY{p}{)}\PY{p}{)}
\end{Verbatim}


    \begin{Verbatim}[commandchars=\\\{\}]
t-statistic: -2.285
p-value: 0.02394

    \end{Verbatim}

    \begin{Verbatim}[commandchars=\\\{\}]
{\color{incolor}In [{\color{incolor}22}]:} \PY{c+c1}{\PYZsh{} males}
         \PY{n+nb}{print}\PY{p}{(}\PY{l+s+s1}{\PYZsq{}}\PY{l+s+s1}{MALES}\PY{l+s+s1}{\PYZsq{}}\PY{p}{)}
         \PY{n}{xbar\PYZus{}m} \PY{o}{=} \PY{n}{np}\PY{o}{.}\PY{n}{mean}\PY{p}{(}\PY{n}{temp\PYZus{}m}\PY{p}{)}
         \PY{n}{s\PYZus{}m} \PY{o}{=} \PY{n}{np}\PY{o}{.}\PY{n}{std}\PY{p}{(}\PY{n}{temp\PYZus{}m}\PY{p}{)}
         \PY{n+nb}{print}\PY{p}{(}\PY{l+s+s1}{\PYZsq{}}\PY{l+s+s1}{sample mean: }\PY{l+s+si}{\PYZob{}\PYZcb{}}\PY{l+s+se}{\PYZbs{}n}\PY{l+s+s1}{sample standard deviation: }\PY{l+s+si}{\PYZob{}\PYZcb{}}\PY{l+s+se}{\PYZbs{}n}\PY{l+s+s1}{\PYZsq{}}\PY{o}{.}\PY{n}{format}\PY{p}{(}\PY{n+nb}{round}\PY{p}{(}\PY{n}{xbar\PYZus{}m}\PY{p}{,} \PY{l+m+mi}{3}\PY{p}{)}\PY{p}{,} \PY{n+nb}{round}\PY{p}{(}\PY{n}{s\PYZus{}m}\PY{p}{,} \PY{l+m+mi}{3}\PY{p}{)}\PY{p}{)}\PY{p}{)}
         
         \PY{c+c1}{\PYZsh{} confidence interval for one draw}
         \PY{n}{ci\PYZus{}low\PYZus{}m}\PY{p}{,} \PY{n}{ci\PYZus{}high\PYZus{}m} \PY{o}{=} \PY{n}{stats}\PY{o}{.}\PY{n}{norm}\PY{o}{.}\PY{n}{interval}\PY{p}{(}\PY{l+m+mf}{0.95}\PY{p}{,} \PY{n}{loc}\PY{o}{=}\PY{n}{xbar\PYZus{}m}\PY{p}{,} \PY{n}{scale}\PY{o}{=}\PY{n}{s\PYZus{}m}\PY{p}{)}
         \PY{n+nb}{print}\PY{p}{(}\PY{l+s+s1}{\PYZsq{}}\PY{l+s+s1}{95}\PY{l+s+si}{\PYZpc{} c}\PY{l+s+s1}{onfidence interval (one draw): }\PY{l+s+si}{\PYZob{}\PYZcb{}}\PY{l+s+s1}{ \PYZhy{} }\PY{l+s+si}{\PYZob{}\PYZcb{}}\PY{l+s+s1}{\PYZsq{}}\PY{o}{.}\PY{n}{format}\PY{p}{(}\PY{n+nb}{round}\PY{p}{(}\PY{n}{ci\PYZus{}low\PYZus{}m}\PY{p}{,} \PY{l+m+mi}{3}\PY{p}{)}\PY{p}{,} \PY{n+nb}{round}\PY{p}{(}\PY{n}{ci\PYZus{}high\PYZus{}m}\PY{p}{,} \PY{l+m+mi}{3}\PY{p}{)}\PY{p}{)}\PY{p}{)}
         
         \PY{c+c1}{\PYZsh{} females}
         \PY{n+nb}{print}\PY{p}{(}\PY{l+s+s1}{\PYZsq{}}\PY{l+s+se}{\PYZbs{}n}\PY{l+s+se}{\PYZbs{}n}\PY{l+s+s1}{FEMALES}\PY{l+s+s1}{\PYZsq{}}\PY{p}{)}
         \PY{n}{xbar\PYZus{}f} \PY{o}{=} \PY{n}{np}\PY{o}{.}\PY{n}{mean}\PY{p}{(}\PY{n}{temp\PYZus{}f}\PY{p}{)}
         \PY{n}{s\PYZus{}f} \PY{o}{=} \PY{n}{np}\PY{o}{.}\PY{n}{std}\PY{p}{(}\PY{n}{temp\PYZus{}f}\PY{p}{)}
         \PY{n+nb}{print}\PY{p}{(}\PY{l+s+s1}{\PYZsq{}}\PY{l+s+s1}{sample mean: }\PY{l+s+si}{\PYZob{}\PYZcb{}}\PY{l+s+se}{\PYZbs{}n}\PY{l+s+s1}{sample standard deviation: }\PY{l+s+si}{\PYZob{}\PYZcb{}}\PY{l+s+se}{\PYZbs{}n}\PY{l+s+s1}{\PYZsq{}}\PY{o}{.}\PY{n}{format}\PY{p}{(}\PY{n+nb}{round}\PY{p}{(}\PY{n}{xbar\PYZus{}f}\PY{p}{,} \PY{l+m+mi}{3}\PY{p}{)}\PY{p}{,} \PY{n+nb}{round}\PY{p}{(}\PY{n}{s\PYZus{}f}\PY{p}{,} \PY{l+m+mi}{3}\PY{p}{)}\PY{p}{)}\PY{p}{)}
         
         \PY{c+c1}{\PYZsh{} confidence interval for one draw}
         \PY{n}{ci\PYZus{}low\PYZus{}f}\PY{p}{,} \PY{n}{ci\PYZus{}high\PYZus{}f} \PY{o}{=} \PY{n}{stats}\PY{o}{.}\PY{n}{norm}\PY{o}{.}\PY{n}{interval}\PY{p}{(}\PY{l+m+mf}{0.95}\PY{p}{,} \PY{n}{loc}\PY{o}{=}\PY{n}{xbar\PYZus{}f}\PY{p}{,} \PY{n}{scale}\PY{o}{=}\PY{n}{s\PYZus{}f}\PY{p}{)}
         \PY{n+nb}{print}\PY{p}{(}\PY{l+s+s1}{\PYZsq{}}\PY{l+s+s1}{95}\PY{l+s+si}{\PYZpc{} c}\PY{l+s+s1}{onfidence interval (one draw): }\PY{l+s+si}{\PYZob{}\PYZcb{}}\PY{l+s+s1}{ \PYZhy{} }\PY{l+s+si}{\PYZob{}\PYZcb{}}\PY{l+s+s1}{\PYZsq{}}\PY{o}{.}\PY{n}{format}\PY{p}{(}\PY{n+nb}{round}\PY{p}{(}\PY{n}{ci\PYZus{}low\PYZus{}f}\PY{p}{,} \PY{l+m+mi}{3}\PY{p}{)}\PY{p}{,} \PY{n+nb}{round}\PY{p}{(}\PY{n}{ci\PYZus{}high\PYZus{}f}\PY{p}{,} \PY{l+m+mi}{3}\PY{p}{)}\PY{p}{)}\PY{p}{)}
\end{Verbatim}


    \begin{Verbatim}[commandchars=\\\{\}]
MALES
sample mean: 98.105
sample standard deviation: 0.693

95\% confidence interval (one draw): 96.746 - 99.464


FEMALES
sample mean: 98.394
sample standard deviation: 0.738

95\% confidence interval (one draw): 96.948 - 99.84

    \end{Verbatim}

    \begin{Verbatim}[commandchars=\\\{\}]
{\color{incolor}In [{\color{incolor}23}]:} \PY{c+c1}{\PYZsh{} bootstrap \PYZhy{} two\PYZhy{}sided Welch\PYZsq{}s t\PYZhy{}test  }
         \PY{n}{size} \PY{o}{=} \PY{l+m+mi}{10000}
         \PY{n}{bs\PYZus{}replicates\PYZus{}m} \PY{o}{=} \PY{n}{np}\PY{o}{.}\PY{n}{empty}\PY{p}{(}\PY{n}{size}\PY{p}{)}
         
         \PY{k}{for} \PY{n}{i} \PY{o+ow}{in} \PY{n+nb}{range}\PY{p}{(}\PY{n}{size}\PY{p}{)}\PY{p}{:}
             \PY{n}{bs\PYZus{}sample\PYZus{}m} \PY{o}{=} \PY{n}{np}\PY{o}{.}\PY{n}{random}\PY{o}{.}\PY{n}{choice}\PY{p}{(}\PY{n}{temp\PYZus{}m}\PY{p}{,} \PY{n+nb}{len}\PY{p}{(}\PY{n}{temp\PYZus{}m}\PY{p}{)}\PY{p}{)}
             \PY{n}{bs\PYZus{}replicates\PYZus{}m}\PY{p}{[}\PY{n}{i}\PY{p}{]} \PY{o}{=} \PY{n}{np}\PY{o}{.}\PY{n}{mean}\PY{p}{(}\PY{n}{bs\PYZus{}sample\PYZus{}m}\PY{p}{)}
             
         \PY{n}{bs\PYZus{}mean\PYZus{}m} \PY{o}{=} \PY{n}{np}\PY{o}{.}\PY{n}{sum}\PY{p}{(}\PY{n}{bs\PYZus{}replicates\PYZus{}m}\PY{p}{)}\PY{o}{/}\PY{n}{size}
         
         \PY{n}{bs\PYZus{}replicates\PYZus{}f} \PY{o}{=} \PY{n}{np}\PY{o}{.}\PY{n}{empty}\PY{p}{(}\PY{n}{size}\PY{p}{)}
         
         \PY{k}{for} \PY{n}{i} \PY{o+ow}{in} \PY{n+nb}{range}\PY{p}{(}\PY{n}{size}\PY{p}{)}\PY{p}{:}
             \PY{n}{bs\PYZus{}sample\PYZus{}f} \PY{o}{=} \PY{n}{np}\PY{o}{.}\PY{n}{random}\PY{o}{.}\PY{n}{choice}\PY{p}{(}\PY{n}{temp\PYZus{}f}\PY{p}{,} \PY{n+nb}{len}\PY{p}{(}\PY{n}{temp\PYZus{}f}\PY{p}{)}\PY{p}{)}
             \PY{n}{bs\PYZus{}replicates\PYZus{}f}\PY{p}{[}\PY{n}{i}\PY{p}{]} \PY{o}{=} \PY{n}{np}\PY{o}{.}\PY{n}{mean}\PY{p}{(}\PY{n}{bs\PYZus{}sample\PYZus{}f}\PY{p}{)}
         
         \PY{n}{bs\PYZus{}mean\PYZus{}f} \PY{o}{=} \PY{n}{np}\PY{o}{.}\PY{n}{sum}\PY{p}{(}\PY{n}{bs\PYZus{}replicates\PYZus{}f}\PY{p}{)}\PY{o}{/}\PY{n}{size}
         
         
         \PY{n}{result} \PY{o}{=} \PY{n}{stats}\PY{o}{.}\PY{n}{ttest\PYZus{}ind}\PY{p}{(}\PY{n}{bs\PYZus{}replicates\PYZus{}f}\PY{p}{,} \PY{n}{bs\PYZus{}replicates\PYZus{}m}\PY{p}{,} \PY{n}{equal\PYZus{}var}\PY{o}{=}\PY{k+kc}{False}\PY{p}{)}
         
         \PY{n+nb}{print}\PY{p}{(}\PY{l+s+s1}{\PYZsq{}}\PY{l+s+s1}{Welch}\PY{l+s+se}{\PYZbs{}\PYZsq{}}\PY{l+s+s1}{s t\PYZhy{}test:}\PY{l+s+se}{\PYZbs{}n}\PY{l+s+s1}{t\PYZhy{}statistic: }\PY{l+s+si}{\PYZob{}:0.5\PYZcb{}}\PY{l+s+se}{\PYZbs{}n}\PY{l+s+s1}{p\PYZhy{}value: }\PY{l+s+si}{\PYZob{}:0.5\PYZcb{}}\PY{l+s+s1}{\PYZsq{}}\PY{o}{.}\PY{n}{format}\PY{p}{(}\PY{n}{result}\PY{p}{[}\PY{l+m+mi}{0}\PY{p}{]}\PY{p}{,} \PY{n}{result}\PY{p}{[}\PY{l+m+mi}{1}\PY{p}{]}\PY{p}{)}\PY{p}{)}
\end{Verbatim}


    \begin{Verbatim}[commandchars=\\\{\}]
Welch's t-test:
t-statistic: 227.09
p-value: 0.0

    \end{Verbatim}

    Analysis:

The mean human body temperature has been proven to be 98.25°, rather
than 98.6°. The sample consists of 130 samples; 65 male and 65 female.

A boxplot (Fig. 6.1) shows that the females' mean body temperature is
slightly higher than the males'.

CDFs (Fig. 6.2) show that both sample sets roughly follow the standard
normal distribution, with the females' temperatures running a little
warmer and with slightly more variation.

The distributions for both samples approximate the standard normal
distribution and both sample sizes are sufficiently large, so the
Central Limit Theorem applies and inferences can be made based on the
properties of the normal distribution.

\[H_0: \bar x_m = \bar x_f\] \[H_A: \bar x_m \neq \bar x_f\]

The null hypothesis is that the mean male body temperature is equal to
the mean female body temperature. An unequal variance \(t\)-test will be
used to test the means of two continuous distributions with unequal
variances. Because the null hypothesis is that the means are equal, a
two-tailed test is required.

The p-values and \(t\)-statistics for both the bootstrap and frequentist
hypothesis tests require that the null hypothesis be rejected and
support the alternate hypothesis that mean male and female body
temperatures are different. A table summarizing the differences between
male and female mean body temperatures is provided below:

Male

Female

\(\bar x\)

98.105

98.394

\(s\)

0.693

0.738

Margin of Error

1.359

1.446

95\% Conf. Int.

(96.746, 99.464)

(96.948, 99.84)


    % Add a bibliography block to the postdoc
    
    
    
    \end{document}
